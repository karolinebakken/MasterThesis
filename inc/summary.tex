\chapter*{Abstract}

As offshore wind energy installations are moving into deeper waters, and further away from shore, floating wind installations are inevitable. The dynamic power cables that deliver electricity from the installation are essential components for this technology. There has been substantial research on the fatigue life on dynamic slender structures in the oil and gas industry, but very little in dynamic power cables applied in offshore wind. Learning more about fatigue of power cables is a key element to advance in offshore floating wind technology. Generally, a power cable consists of conductors with an assembly of wires helically stranded in layers around a core, causing contact between each other as well as between layers in each conductor, making the cables vulnerable for several different fatigue mechanisms.\\\\
This topic was investigated further by performing a global and local analysis. The floating wind turbine OO-Star was chosen as a basis for a case study. OO-Star is a design by Dr. Techn. Olav Olsen, and is participating in the project Lifes50+. West of Barra, Scotland was chosen as the location, and the cable design was developed in close dialogue with Professor Svein Sævik. The global model was built in SIMA RIFLEX, and consisted of the whole dynamic part of the cable. The cable was attached to the sea floor, and a point on the free surface with motions like the wind turbine floater due to given transfer functions provided by Dr. Techn. Olav Olsen. No compression and a maximum curvature not to be exceeded laid the basis for the configuration of the cable. The focus of this study was to get an insight into the local contact effects of the cross-section, rather than to perform a full fatigue analysis. For this reason, only three different wind conditions and positions were included. The local model was created in Bflex. It was assumed that the most severe fatigue damage would occur at the cable hang-off, hence only the upper part of the cable was modeled. The local model consisted of a rigid pipe, a bend stiffener, and a cable with several layers and contact elements between them.\\\\ The global analyses were performed by running each sea state in the scatter diagram for West of Barra for one hour each and calculating the angle between the cable and vessel, and the tension in the upper element. Rainflow Counting was used to count cycles in each angle class, and the classes were rearranged into 15 cases with angle range and corresponding tension. The 15 cases served as the input for the local analysis were each dynamic angle and tension pair range were analyzed together for one cycle. The curvature and tension in each case were used to calculate the stress range for each case in each layer of the conductor. Eventually, an appropriate SN-curve was used to estimate the fatigue life.  \\\\
The fatigue life of the dynamic power cable was estimated to be 267.37 years, and further analyses showed that the effect of friction between conductors and between layer in each conductor played a prominent role in the fatigue life.  