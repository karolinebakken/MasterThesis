\chapter*{Abstract}

As offshore wind energy installations are moving into deeper waters, and further away from shore, floating wind installations are necessary. The dynamic power cable that will deliver the electricity from the installation is an important component for this technology. There has been a lot of research on the fatigue life on dynamic slender structures in the oil and gas industry, but very little in dynamic power cables applied in offshore wind. Learning more on fatigue of power cables is a key component in order to advance in offshore floating wind technology. Generally, a power cable consists of conductors with an assembly of wires helically stranded in layers around a core, causing contact between each other as well as between layers in each conductor, making the cables vulnerable for several different friction mechanisms.\\\\ This topic was investigated further by performing a global and a local analysis. The floating wind turbine OO-Star was chosen as a case study. This is a design done by Dr. Techn. Olav Olsen, and is participating in the project Lifes50+. West of Barra, Scotland was chosen as location, and the cable design was developed in close dialogue with Professor Svein Sævik. SIMA RIFLEX was used for the global model, where the whole cable was modelled with attachment to the sea floor, and a point behaving like the wind turbine floater due to given transfer functions provided by Dr. Techn. Olav Olsen. The configuration of the cables was based on no compression in the cable and a max curvature that could not be exceeded in the static or dynamic analysis for neither wind condition or position. As a simplification, three different wind conditions and positions were included.  The local model was created in BFLEX, and only the upper part of the cable was modelled as it was assumed that this is were the largest fatigue damage is located. The local model consisted of a stiff pipe, a bend stiffener and a cable with several layers. \newline 
\newline
The global analysis was performed by running each sea state in the scatter diagram for West of Barra for one hour each, and calculating the angle between the cable and vessel, and the tension in the upper element. Rainflow counting was used to count cycles in each angle class, and the classes were rearranged into 15 cases with angle range and corresponding tension. This was used as input in the local analysis were each angle and tension range were analyzed together for one cycle. The curvature and tension in each case was used to calculate the stress range for each case in each layer of the conductor, and fatigue life was estimated by the use of an appropriate SN-curve.  \\\\
The fatigue life of the dynamic power cable was estimated to 640.28 years, and analyses showed that the effect of friction between conductors and between layer in each conductor played a big role in the fatigue life. \\\\
Also included in the Master Thesis are all the necessary theory necessary to execute the procedure described above. This includes theory about wind turbines, power cables, fatigue and the numeric theory behind the software used to create the two models. 

