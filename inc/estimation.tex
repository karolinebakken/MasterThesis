\chapter{Analyses Methodology}
\label{chap:estimation}
In this section an overview of how the global and local analyses were performed and the estimation of the fatigue life of the dynamic power cable will be presented, as shown in Figure \ref{fig:aanaalg}. 
\begin{figure}[H]
\centering
\includegraphics[scale=0.73]{figures/anaaalg}
\caption[Algorithm for calculation of fatigue life ]{Algorithm for calculation of fatigue life  }
 \label{fig:aanaalg}
\end{figure}
\section{Global Analysis}
First, the global analysis was carried out. The global model described in Section \ref{sec:globmod} was used, and the 160 sea states presented in Figure \ref{fig:scatn} was applied to the model for 1 hour each, with Far weather conditions and vessel position as described in Sections \ref{sec:current} and \ref{sec:globmod}. The following three quantities from the analyses were stored in separate files for each time step:
\begin{itemize}
    \item The  tension in the upper element of the cable.
    \item The displacement of the dummy node on the vessel in x, y and z direction.
    \item The  displacement of the super node at the cable hang off, and first  node in the top the cable in x, y and z direction.
\end{itemize}

\subsection{Post Processing of Global Analyses Results}
The results from the global analyses determined the input for the local analyses. Time series of the displacements of the three nodes seen in Figure \ref{fig:angle} was used to calculate the angle ($\theta$) between the vessel and the cable by dot product: 

\begin{equation}
    cos(\theta) = \frac{\boldsymbol{a \cdot b}}{|\boldsymbol{a}||\boldsymbol{b}| }
\end{equation}

Where $\boldsymbol{a}$ and $\boldsymbol{b}$ are the vectors of the dummy element and the first element of the cable. 

\begin{figure}[H]
\centering
\includegraphics[scale=0.5]{figures/angle}
\caption[Angle between vessel and cable ]{Angle between vessel and cable  }
 \label{fig:angle}
\end{figure}

\noindent The calculated angle and tension were stored as time series for each sea state. Examples of the time series for angle and tension as plots can be seen in Figure \ref{fig:angleex} and Figure \ref{fig:tensex}

\begin{figure}[H]
\subfloat[Time series for angle between vessel and cable \label{fig:angleex}]
  {\includegraphics[width=.45\linewidth]{figures/angleex}}\hfill
\subfloat[Time series for tension in the upper element \label{fig:tensex}]
  {\includegraphics[width=.45\linewidth]{figures/tensex}}\hfill
\caption{Example of time series for angle and tension}
\label{fig:timeex}
\end{figure}

section{Rainflow Counting}
 Rainflow counting was performed on the time series of the angle to determine the number of cycles for each angle range. The Rainflow Counting was performed in Python, and the algorithm is explained in Section \ref{sec:rainflow}. The Rainflow counting in Python takes in a one dimensional array of numbers and gives out a matrix where the first column is the class and the second is the number of cycles in each class. The counts from each sea state were scaled after occurrence in the scatter diagram in Figure \ref{fig:scatn} to apply for a year according to:

\begin{equation}
    n_{cycle,year}=n_{cycle,hour} \frac{n_{seastate}}{N_{seastate}} \cdot 365 \cdot 24 
\end{equation}

\noindent Where $n_{cycle,year}$ is the number of cycles of in an angle class in a whole year, $n_{cycle,hour}$ is the number of cycles in an angle class for an hour, calculated by the global analyses, $n_{seastate}$ is the number of observations of a certain sea state according to the scatter diagram in Figure \ref{fig:scatn} and $N_{seastate}$ is the total number of sea states in the scatter diagram in Figure \ref{fig:scatn}.\newline
\newline 
\noindent The results from the Rainflow Counting can be seen in Figure \ref{fig:initialcyc}.  

\begin{figure}[H]
\centering
\includegraphics[scale=0.9]{figures/initialcyc}
\caption[Number of cycles in each class]{Number of cycles in each class}
 \label{fig:initialcyc}
\end{figure}

\noindent In Figure \ref{fig:initialcyc}, the angle range classes were of equal length of 0.3. It is not possible to see the classes on the x-axis due to the large amount of classes. It is is clear from Figure \ref{fig:initialcyc} that the majority of the cycles are happening in the first angle range classes, and decreasing as the angle range increases.\newline
\newline
The damage from each range was calculated according to Miner Palmgren expression in Equation \ref{eq:MP}, where N was calculated from Equation \ref{eq:sn}. This gave the following expression for the damage from each angel class: 
\begin{equation}
    d_i  = \frac{n_i}{N_i}
\end{equation}
\begin{equation}
    d_i=\frac{n_i}{\frac{c}{\Delta \sigma ^m}}
\end{equation}
Where $d_i$ is the damage from the cycles in angel class i, $n_i$ is the number of cycles in angle class i, $N_i$ is the number of cycles until failure for angle class i, c and m are from Equation \ref{eq:sn}, c=4 and m=2.88e25 for this case,  $\Delta \sigma$ is the stress range that relates to the angle range as: $\Delta \sigma_i = a \Delta \theta_i$ where a is an arbitrary constant.\newline
\newline 
The damage from the cycles in the different angle classes can be seen in Figure \ref{fig:initialdam}. It is clear that the middle classes gave the largest damage. This is because the small classes have high cycle count but very low damage pr cycle, and the high classes have high damage pr cycle, but very few cycles. 

\begin{figure}[H]
\centering
\includegraphics[scale=0.9]{figures/initialdam}
\caption[Damage for each angle class]{Damage for each angle class}
 \label{fig:initialdam}
\end{figure}
\noindent In order to get good results for the local analyses, the classes needed to be merged and rearranged so that the damage was approximately the same over each class. The results of this rearrangement can be seen in Figure \ref{fig:initialdam}

\begin{figure}[H]
\centering
\includegraphics[scale=0.9]{figures/newdam}
\caption[Damage for each angle class, rearranged]{Damage for each angle class, rearranged}
 \label{fig:newdam}
\end{figure}

The rearrangement of the classes led to the cycle distribution shown in Figure \ref{fig:newcyc} and Table \ref{table:angleclass}:

\begin{figure}[H]
\centering
\includegraphics[scale=1]{figures/newcyc}
\caption[Distribution of cycles for each angle class, rearranged for equal damage]{Distribution of cycles for each angle class, rearranged for equal damage}
 \label{fig:newcyc}
\end{figure}

\begin{table} [H]
\centering
\begin{tabular}{ |c|c|}
\hline
Angle range class [deg] & Number of cycles \\
 \hline
 \hline
0.0 - 2.7 & 6930399.8\\

2.8 - 3.3 & 571967.9\\
 
3.4 - 3.6 & 237278.5 \\
 
3.7 - 3.9& 219492.0  \\

4.0 - 4.2& 198970.4  \\

4.3  - 4.5 & 154052.1  \\

4.6 - 4.8 & 130555.0 \\

4.9 - 5.1 & 93955.7 \\

5.2 - 5.4 & 68946.3 \\

5.5 - 5.7 & 46839.2 \\

5.8 - 6.3 & 54651.8 \\

6.4 - 6.9 & 24288.7 \\

7.0 - 8.4 & 16385.8 \\

8.5 - 18.0 & 3355.4 \\

18.1 - 23.4 & 139.3  \\

 \hline
\end{tabular}
\caption{Number of cycles in each angle range}
\label{table:angleclass}
\end{table} 
\noindent It is important to note that the number of cycles for each class are not just half and whole cycles, as the number of cycles have been scaled for occurrence over a year.\newline 
\newline

\section{Local Analysis}
The local analyses were performed by using the results from the global analyses. As the local model was pinned in both ends, both ends of the model experienced rotation when a dynamic angle was applied to the model. This can be seen in Figure \ref{fig:anglecorr}. $\theta_1$ was equal to the applied angle, but the total angle experienced by the model was:
\begin{equation}
    \theta_{tot}=\theta_1 + \theta_2
\end{equation}
Where $\theta_{tot}$ is the total angle experienced by the local model and $\theta_1$ and $\theta_2$ are illustrated in Figure \ref{fig:anglecorr}, and shown in the real model in Figure \ref{fig:anglecorrre}.

\begin{figure}[H]
\centering
\includegraphics[scale=0.75]{figures/anglecorr1}
\caption[Initial and deformed configuration of local model]{Illustration of initial and deformed configuration of local model}
 \label{fig:anglecorr1}
\end{figure}

\begin{figure}[H]
\centering
\includegraphics[scale=0.75]{figures/anglecorrre}
\caption[Deformed local model]{Deformed local model}
 \label{fig:anglecorrre}
\end{figure}

\noindent Several analyses were done to create a relation between the applied angle and the total angle. The relationship was found to be linear with a coefficient of determination ($R^2$) of 0.9978. The linear relationship can be seen in Figure \ref{fig:anglerel}, with the following expression:

\begin{equation}
    \theta_{tot}=1.2694\theta_{appl.}
\end{equation}
Where $\theta_{tot}$ is the total angle experienced by the local model and $\theta_{appl.}$ is the angle applied by BFLEX.
\begin{figure}[H]
\centering
\includegraphics[scale=0.8]{figures/anglerel}
\caption[Relationship between applied angle and total angle]{Relationship between applied angle and total angle}
 \label{fig:anglerel}
\end{figure}

\noindent It was decided in line with Professor Svein Sævik's recommendations that the angle would be the master over the tension, meaning that for the local analyses, the cycle count for the angle would be used with the corresponding tension. Even though the trends in the angle time series and the tension time series were the same, the high angles and tensions did not occur at the exact same time steps, so that linear regression showed no correlation. It was decided that the max tension would correspond to the max angle, and the min tension would correspond to the min angle, as this would be a conservative approach.  Linear interpolation was used to find the tension to the corresponding angle. The max and min tension registered at the top of the cable for all sea states was 40.69 kN and 9.32 kN respectively. \newline
\newline
\noindent The 15 angle range classes in Table \ref{table:angleclass} was turned into the 15 BFLEX cases displayed in Appendix \ref{table:loadcase}. Bend stiffeners are normally initially mounted in a stress free configuration. For this study it was though that the stress free configuration for the bend stiffener was in Neutral position wiht no current. This corresponds to an angle of -2.32$^\circ$ found from the SIMA RIFLEX, with zero degrees being if the cable was hanging straight down and positive angle is counter clockwise. The dynamic analyses were performed in the far position with current, where the static position of the bend stiffener was 15.78$^\circ$ relative to the stress free configuration due to the strong current as can be seen in Figure \ref{fig:statconc}. The total mean angle was therefore 18.16$^\circ$, and the max and min angle for each angle range were added and subtracted to the mean angle, so that the cable would bend around this angle. As the angles needed to be corrected due to the boundary conditions of the local model as seen in Figure \ref{fig:anglerel}, the min angle and max angle in Table \ref{table:loadcase} are the corrected min and max angle, and thus the angle input in the local analyses. The minimum dynamic tension and maximum dynamic tension were found through interpolation, and the mean tension represents the static tension at the top of the cable after current has been applied in far position. The analyses were run by a Cygwin script, making it easy to change variables in each case. 


\begin{table} [H]
\centering
\begin{tabular}{ |c|c|c|c|c|c|}
\hline
	Case Nr. & Max Angle [deg] & Min angle [deg] & Mean tens.[kN] & Max tens.[kN]  & Min tens.[kN]   \\
 \hline
 \hline
	1 & 16.43 & 12.18 & 13.7 & 25.07 & 11.50   \\ 
	2 &  16.91 & 11.71 & 13.7 & 25.52 & 11.44   \\
	3 &  17.14 & 11.47 & 13.7 & 25.75 & 11.41   \\ 
	4 &  17.38 & 11.23 & 13.7 & 25.97 & 11.37  \\ 
	5 &  17.61 & 11.00 & 13.7 & 26.20 & 11.34  \\ 
	6 &  17.85 & 10.76 & 13.7 & 26.42 & 11.31  \\ 
	7 &  18.09 & 10.52 & 13.7 & 26.65 & 11.28   \\ 
	8 &  18.32 & 10.29 & 13.7 & 26.88 & 11.25  \\ 
	9 &  18.56 & 10.05 & 13.7 & 27.10 & 11.22  \\ 
	10 &  18.80 & 9.82 & 13.7 & 27.33 & 11.18  \\ 
	11 & 19.27 & 9.34 & 13.7 & 27.78 & 11.12  \\ 
	12 &  19.74 & 8.87 & 13.7 & 28.24 & 11.06  \\ 
	13 &  20.92 & 7.69 & 13.7 & 29.37 & 10.90  \\ 
	14 &  28.49 & 0.13 & 13.7 & 36.62 & 9.89  \\ 
	15 &  32.74 & -4.13 & 13.7 & 40.69 & 9.32 \\ 
 \hline
\end{tabular}
\caption{Parameters for load cases in the Local Analyses}
\label{table:loadcase}
\end{table} 

\section{Calculation of Fatigue Life}
\label{sec:fatiguelife}
Tension and curvature over the length of the conductor for each of the 15 cases were studied, and the plots can be seen in Appendix \ref{appendix:C}. The fatigue life was calculated for each layer at two points on the conductors: 
\begin{itemize}
\item At max curvature range (x=0.900m)
\item At max tension range  (x=0.625m)
\end{itemize}
Theses two points did not coincide due to a phase shift caused by friction. The the calculation methodology was the same for both points.  \\\\
The stress range on wire level was calculated for each layer in each case from the Equations \ref{eq:stressvariation} to \ref{eq:stressvariation2} presented in Section \ref{sec:fatcop}. Lay angles were $\alpha_2=-4.1^\circ$ and $\alpha_3=+7.3^\circ$ for the innermost layer and outer layer respectively, taken from \cite{Nasution2013}. Friction coefficient was found to be $\mu=0.2$ from \cite{NASUTION2014}. The stress concentration factor (SCF) for the innermost layer and the outer layer were set to 1.058 and 1.238 respectively, also from \cite{NASUTION2014}. \\\\
The number of cycles until failure ($N_i$) was calculated from Equation \ref{eq:sn}. Damage from each case in one year was calculated according to Miner-Palmgren sum in Equation \ref{eq:MP}, using the number of cycles found by rainflow counting in Table \ref{table:angleclass}. By inverting the damage caused in one year, the fatigue life in year was calculated with DFF=10.0. The results underwent mean stress correction according to Söderberg correction described in Section \ref{sec:meanstress}. 

\section{Sensitivity Studies and Study of Friction Mechanisms}
To get an increased understanding of the contact mechanisms in the conductors, several additional studies were performed with the results from the global and local analyses. The following studies were done:
\begin{itemize}
    \item Study of sensitivity by using mean and min tension in Equation \ref{eq:stressvariation2}
    \item Study of effect of contact between conductors
    \item Study of effect of contact between layers in conductors
\end{itemize}
\subsection{Study of Sensitivity by using Mean and Min Tension in Equation \ref{eq:stressvariation2}}
\label{sec:min}
When calculating the fatigue life as explained in Section \ref{sec:fatiguelife}, one of the variables in Equation \ref{eq:stressvariation2} is tension (T). (\cite{s300}) does not specify which tension to be used, the min tension, mean tension or max tension. However, it is obvious that the most conservative approach is to use the max tension, and this was used in the calculations in Section \ref{sec:fatiguelife}. As a sensitivity study, the fatigue life was also calculated using the mean and min tension in Equation \ref{eq:stressvariation2} to see what effect this would have on the fatigue life. 
\subsection{Study of Effect of Contact Between Conductors}
The three helical conductors were ordered as a triangle, with the HCONT454 element describing the hoop contact between them, as shown in Figure \ref{fig:HCONT4541}. The effect of this contact was investigated by looking at the plot of tension in the conductors over the length of the cable at max angel and min angle for each case. The effect of the contact between the conductors could be seen in the tension plots as the difference between the max tension range($\Delta T_{max}$) illustrated by the green arrow, and the tension range from applied tension ($\Delta T_T$) at the end of the model, illustrated by the yellow arrow. 

\begin{figure}[H]
\centering
\includegraphics[scale=0.7]{figures/confric.PNG}
\caption[$\; \:$ $\Delta T_f$]{$\Delta T_f$}
 \label{fig:condfric}
\end{figure}
The contribution to the tension range from the contact friction could be calculated as:
\begin{equation}
    \Delta T_f = \Delta T_{max} - \Delta T_T
\end{equation}
Where $\sigma T_f$ is the contribution to the tension range form contact between the conductors, $\Delta T_{max}$ is the maximum tension range and $\Delta T_T$ is the tension range from applied tension to the local model.\\\\
$\Delta T_f$ was subtracted from $\Delta T$ in Equation \ref{eq:sigmaT}, and the fatigue life was calculated as described in Section \ref{sec:fatiguelife}.

\subsection{Study of Effect of Contact Between Layers in Conductors}
The friction between the layers in the conductor was included in the fatigue life calculation by the friction term: $\Delta \sigma_f^i$ in Equation \ref{eq:stressvariationred}. By excluding this term, effect was removed, and Equation \ref{eq:stressvariationred} for the stress range was reduced to:
\begin{equation}
    \Delta \sigma=\Delta \sigma_T \cdot SCF + \Delta \sigma_{nc}
\end{equation}
The futher calculations were performed as in Section \ref{sec:fatiguelife}.




