\chapter{Introduction}
\label{chap:introduction}
\section{Motivation and Background}
As the world strives towards greener energy, different renewable energy alternatives are being explored. Wind energy has been utilized by man for over thousands of years, and in the recent years become one of the more mature renewable energies. As the areas on land that are fit for wind turbines are scarce, and the public is disliking the noise and visibility of the wind turbines, technology is pushing towards offshore wind parks. Moving the wind parks offshore is not only solving the problems with complaints from the public, but wind speed tends to increase with distance from shore. In the last years, there has been an increase in scale, capacity and distance from shore each year for offshore wind installations. Moving further from shore often implies moving into deeper waters, making floating wind turbines the most plausible technology. Floating wind turbines are similar to semi-submersible oil platforms, but will have very different motions, as geometry and wind loads are significantly different. A lot of research has been done in terms of estimating the lifetime for flexible risers and umbilicals in relations to fatigue for the oil and gas industry. For floating wind turbines, there is significantly less research done in terms of the lifetime of dynamic flexible cables. As the power cables that are connected to floating structures will be subjected to oscillations due to movements of the vessel in waves, fatigue strength needs to be verified for design. A conductor usually consists of several conductors with layers of copper wires in each conductor. The wires are stranded helically around a core wire, and this causes contact both between the layer and within each layer. This makes the cable vulnerable for fretting fatigue in addition to fatigue due to the cyclic variation in curvature and tension due to the vessels movements in the waves.  \cite{Savik2014}. \cite{Karlsen2010} explains that neither the fatigue properties, nor the methods to establish SN-data of copper conductors are established by the cable or offshore industry.  The power cable technology and the lifetime of power cables are some of the most important limiting factors of moving wind floating wind turbines further away from the shore and into deeper waters.   Research is needed in order to continue the development of floating wind turbines in the future, \cite{Thies2012}.
\newline
\newline
Lifes50+ is a project financed by the Horizon2020 program where the goal is to develop cost-effective floating solutions for 10MW wind turbines, \cite{Horizon2010}. The project consists of 4 different designs and 3 different locations. One of the designs in the project is the OO-Star developed by Dr. Techn. Olav Olsen shown in Figure \ref{fig:oostarintro}. This was considered an appropriate case study to investigate the lifetime of dynamic power cables further.

\begin{figure}[H]
\centering
\includegraphics[scale=0.5]{figures/oostar}
\caption[$\; \:$Dr. Techn. Olav Olsen's OO-Star]{Dr. Techn. Olav Olsen's OO-Star \cite{Lifes50+D4.2} }
 \label{fig:oostarintro}
\end{figure}


\section{Literature Review}
A literature search was done prior to the project thesis. \cite{Feld1995} looked at the applied loads and responses of the metallic elements for subsea configurations. \cite{Alani1997Eoma} studied the effect of mean axial load on axial fatigue life of spiral strands, and concluded that increasing lay angle had a significant negative effect on endurance limit for both the outer and wires in a spiral strand. They also looked at the effect of including Gerber or Goodman correction for the mean axial load, and found that the corrections did not give any significant improvement on the fit.  \cite{Chien2004} discussed different cable designs for 250MW offshore power transmission and concluded that they do not behave like umbilicals dynamically, and that harmonics do occur during transmission. \cite{Raoof2008} looked at axial fatigue design of sheathed spiral strands in deep water applications, and concluded that (among other things) that the fatigue life of stranded spirals is significantly shorter at the end termination than in free-field. \cite{Karlsen2010} presented a test method for simulating fatigue of a dynamic power cable that included the effects of fretting, creep and friction. \cite{Nasution2013} investigated fatigue on 95mm$^2$ copper conductors experimentally and by the use of FEM. Single wires from different layers were tested in tension and the full cross-section was tested in bending. The full cross-section showed lower fatigue strength that the individual wires and trellis points were particularly vulnerable, as cracks initiated from these points. It was suggested that the difference in results between the testing and the FEM analyses was due to the surface irregularities in the wires from the packaging of the conductor. \cite{NASUTION2014} did similar tests and concluded that all first fatigue failures in tension-tension of the full cross-section happened in the outer layers of the conductors, in the thinnest parts of the wires, that is, the trellis points. For tension-bending mode, all the failures occurred in the inner layer of the conductor. It was concluded that this indicated that the effect of friction between layers plays an important role in the lifetime of the cross-section. The report also comments that as the longitudinal stress governs the fatigue performance, beam elements can be used with good results in analyses. \cite{s300} did experimental and FEM analysis of fatigue strength with 300mm$^2$ conductors. This study concluded with that in terms of maximum stress, individual wires from 95mm$^2$  and 300mm$^2$ wires fall in a common scatter band, and that lubricated conductors have longer fatigue life than unlubricated conductors. The study also supports the previous conclusions that crack initiation starts from trellis points. In this article, an analytic method to calculate the stress variation in the individual wires of the conductor was developed. \cite{Taninok2017} looked at dynamic cable systems for 2MW and 100kW floating offshore wind installations and concluded that their proposed cable profile absorbed the floater behavior so that there was no motion in the cable at the bottom hang off.  

\section{State of the Art}
\subsection{Offshore Wind Turbines}
According to ISSC 2018 committee V.4: OFFSHORE RENEWABLE ENERGY: (\cite{Gao2018}): "Offshore wind is by far the most developed technology and
promising cost reduction has been achieved in the last few years", in terms of offshore renewable technologies. Figure \ref{fig:sit17} shows the development of offshore wind capacity until 2017. 

\begin{figure}[H]
\centering
\includegraphics[scale=0.8]{figures/sit17}
\caption[$\; \:$Cumulative Offshore Wind Capacity in 2017]{Cumulative Offshore Wind Capacity in 2017 \cite{GWEC2018}}
 \label{fig:sit17}
\end{figure}

\noindent The majority of the offshore wind farms are located in Europe. Figure \ref{fig:world} shows the current market situation of offshore floating wind turbines. As can be seen, it is mainly Europe, USA, and China that are represented. 


\begin{figure}[H]
\centering
\includegraphics[scale=0.54]{figures/world}
\caption[$\; \:$Current market situation of offshore floating wind turbines]{Current market situation of offshore floating wind turbines \cite{Gao2018}}
 \label{fig:world}
\end{figure}

\noindent In the recent years, installations have moved further from shore, into deeper waters and with large farm configurations and larger turbines.  Longer blades will give advantages in terms of overall cost, but also challenges for installation, \cite{Gao2018}.  Figure \ref{fig:diameter} shows the development of power capacity for the last two centuries. The average size in 2017 was 5.9 MW. That is an increase of 23\% from 2016 according to \cite{we2018}.

\begin{figure}[H]
\centering
\includegraphics[scale=0.7]{figures/diameter}
\caption[$\; \:$Development of turbine size]{Development of turbine size. \cite{Deigen2018}}
 \label{fig:diameter}
\end{figure}

 \noindent In 2016, the first 8 MW turbine was installed in the Irish Sea in the UK. In Norway, Statoil made a step towards commercialization of floating wind farms by installing the first floating wind farm with 5 6MW spar wind turbines called Hywind, outside of Scotland. Hywind can be used at depths up to 800m, and enable offshore wind energy installations for areas that have been unavailable until now \cite{Equinor2018}. Short-term plans for the offshore wind industry include the installation of two small floating wind farms in the US as well as prototype testing for exiting prototypes in Norway, Portugal, and Japan, and planned prototype development in Japan, France, and Germany.  Two large wind farms are planned on the east coast of the US, one 120 MW farm outside of Maryland, and one 90 MW on the coast of New York. \cite{Gao2018}. \newline
 \newline
 \cite{Bailey2014} states that "Commercialization of floating wind farms is
anticipated between 2020 and 2025." In 2014 the European Union set a legally binding target that 27\% of the energy consumption are to come from renewable energy sources in 2030. \cite{EWEA2015}  presents three different scenarios on the development of offshore wind energy by 2030, and in the central scenario, it is suggested that 66 GW come from offshore wind. To achieve this goal, there needs to be an annual average increase of 15\%. \cite{Gao2018} states that this is probably possible as we have seen an increase of 25\%-30\% annually the recent years. US Department of Energy has a goal of 86 GW of energy provided from offshore wind by 2050. \cite{windus2016}. 
\subsection{Power Cables}
The trend over the years has been that offshore wind installations move further away from shore. This development can be seen in Figure \ref{fig:distshore}. 

\begin{figure}[H]
\centering
\includegraphics[scale=0.9]{figures/distshore}
\caption[$\; \:$Distance to shore ]{How distance to shore have increased in recent years \cite{Make2016}}
 \label{fig:distshore}
\end{figure}

\noindent \cite{srinil2016} states that traditionally, the w shape configuration has been used for inter-array cabling, and the lazy wave has been used for export cables, see Figure \ref{fig:cableconfig}. However, according to \cite{ds2010}, the lazy wave configuration is being considered for inter-array in newer projects such as the Statoil's Hywind project, and the Fukushima Floating Offshore Wind
Farm Demonstration Project, \cite{yagihashi2015dynamic}. Inter-array cables are usually three-core copper conductors, armoured with steel wires with insulation around the conductors. 33kV is usually the standard for offshore cables, however, 66kV are under development \cite{srinil2016}. 


\begin{figure}[H]
\subfloat[Traditional W-shape for inter array cable \label{fig:dc}]
  {\includegraphics[width=.45\linewidth]{figures/wshape}}\hfill
\subfloat[Lazy wave shape, traditionally only used for export cables \label{fig:ac}]
  {\includegraphics[width=.45\linewidth]{figures/lw}}\hfill
\caption[$\; \:$Cable configurations]{Different cable configurations, \cite{ds2010}}
\label{fig:cableconfig}
\end{figure}

\section{Objective}
The objective of this Master Thesis is to estimate the lifetime of a dynamic power cable applied in offshore wind farms. Theory and software used and developed for the oil and gas industry, are applied in the relatively new and upcoming field of ocean renewable energy.\\\\
The focus of this study will be to estimate the fatigue life of a dynamic power cable at a chosen location through global and local analyses with the use of well known fatigue calculation methods such as Rainflow counting, SN-curve and Miner-Palmgren.  In addition, the study focuses on getting an insight in the mechanisms affecting the fatigue life of the cable down to fibre level in the power conductors. \\\\
This master thesis is a continuation of a project thesis that was executed in the fall of 2018.\\\\
The main objectives of this study are based on the Master Description attached earlier in this report. However, as the study was conducted, several challenges were encountered. All choices and and changes to the original plan was made in close dialogue with the supervisor for the project, Professor Svein Sævik. The main objectives are:
\begin{enumerate}
    \item Perform a literature study and acquire the necessary theory background on all topics relevant for the Master Thesis, including but not limited to global and local analyses, offshore wind turbines, cable technology and fatigue.
    \item Choose and establish a case scenario regarding wind turbine floater design, location, environmental conditions, cable design and SN-fatigue data
    \item Create global model in SIMA RIFLEX and determine cable configuration. 
    \item Create a local model in Bflex
    \item Perform global and local analyses and process the results to estimate the fatigue life of the dynamic power cable.
    \item Do a sensitivity study by comparing results with layer friction included and not included, and determine the importance on including friction between layers in fatigue analyses of dynamic power cables.  
\end{enumerate} 

\section{Contribution}
The main contribution from this this study is the investigation of fatigue life of the cable down to fibre level. The effect of friction between wires in the same layer and between layers have been studied and compare to not including this effect. It turns out that.... \todo{INKLUDER RESULTATER}. This has been investigated by performing global and local analyses on a specified case scenario. 
\section{Master Thesis Structure}
\begin{itemize}
    \item Chapter 1 contains the introduction with literature review, state of the art and objective for the project thesis.
     \item Chapter 2 is the system theory for wind turbines, offshore wind turbines and power cables.
      \item Chapter 3 goes through the basic theory of wave-induced fatigue. 
      \item The case scenario that will be investigated is presented in Chapter 4
      \item Chapter 5 describes the theory behind the two numeric models
      \item In chapter 6, the modeling methodology will be explained.
      \item In Chapter 7, goes through the analyses methodology
      \item Chapter 8 presents the results and discussion of the results
      \item Chapter 9 is the conclusion and suggestions for further work
\end{itemize}

