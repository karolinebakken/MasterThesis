\chapter{Results and Discussion}
\label{chap:discussion}
 In this chapter, several aspects of the Master Thesis will be discussed. The topic of fatigue of dynamic power cables applied offshore wind farms will be discussed based on the literature review and theory section. The global and local model will be discussed, as well as the results from the analyses. Simplifications regarding the models and analyses will also be discussed. 
 \section{Discussion of Case Study}
 \subsection{Floating Wind Turbine}
The OO-Star from Dr. Techn. Olav Olsen was chosen as the case study for floating wind turbine due to the previous familiarity with the project for the author, through a summer internship. OO-Star is part of a research project called Lifes50+ where different floating wind turbine designs and locations will be investigated. OO-Star it not fully developed, and has not yet been constructed. An advantage of choosing this concept was that there is a lot of information available on the website of Lifes50+. Another option could have been to use the Hywind concept that is already installed at the Hywind Scotland Project, or one of the other concepts in the Lifes50+ project.
\subsection{Location and Environmental Conditions}
The location of the case study was chosen to be West of Barra. This is the location in the project with the deepest water and with the most challenging weather conditions. It was thought that these factors would make the investigation of the lifetime of the power cable the most interesting in terms of fatigue. However, as the work in this study was carried out, it turned out that the wave conditions at this location was slightly too challenging, as several of the most extreme sea stats was impossible to analyze due to resonance. The solution became to move the most extreme observed sea states to sea states with smaller Tp, but still keeping the total number of observations. This way, severe resonance and extreme response was avoided. Luckily, the number of observations for the sea states that had to be modified was very low, giving little effect to the total fatigue life.\\\\
The wind conditions was significantly simplified, and only taken into concideration through the current conditions and the three different positions of the floater, Far, Neutral and Near. The offsets for the different positions were calculated very roughly and a linear system was assumed. The offsets in each position could have been studied more in depth. The scatter diagram for the wind conditions at West of Barra is available, and could have been used to simulate the wind conditions at the location more realistically, however this would make the model and the analyses more complex.


\subsection{Cable}
The cable cross section was based on guidance from Professor Svein Sævik, who has extensive experience in the field. Another approach could have been to contact a subsea cable manufacturer and either use one of their designs.  However, the majority of dynamic power cables are designed specifically for its use, and standardized models are rare. The cable proved to be some what too light weight, as the cable had large response to the vessel movements. To remedy this, 6kg of lead was added to the cross section in the space between the conductors and the inner protective sheath. This improved the inertia of the cable, but one could discuss weather double armouring should be added to the cable to increase the inertia further. Price of steel would have had to be taken into consideration for such a discussion.
\section{Discussion of Modeling Methodology}
\subsection{Global Model}
 The global model was modelled in SIMA RIFLEX. The model consists of the total cable configuration and is attached to a point a calculated distance from the center of gravity of the OO-Star. The point have its motions in accordance to the support structure of the OO-Star due to the transfer functions provided by Dr. Techn. Olav Olsen. The configuration is the characteristic lazy wave, which is simple to model, but hard to ensure that the touchdown is stable. Usually this configuration has an thether on the bottom to keep the touchdown stable, but the area of interest for this study was the cable hang off, so very little attention was paid to the bottom touchdown. The main design criteria for the configuration of the global model was that the max curvature could not be exceeded anywhere in the cable. The configuration was determined through testing and failing, and this process could have been done in a more systematic way, and the cable configuration could probably be optimized.
\subsection{Local Model}
The local model was created in BFLEX and was modelled from the inner layer and outwards. Only the very upper part of the cable was modelled as this is considered to be the area with the most fatigue damage. The friction models used to describe the contact between the layers could have been explored more and other alternative friction models could have been considered. In reality, a conductor with area 95mm$^2$ consists of 19 wires. These were not modelled, and instead equations developed by \cite{s300} was used to include the effect of the wires of the conductor. The armouring layers was only modelled with 4 wires in each layer, but this was accounted for by scaling factor. \\\\ The contact element used to describe the contact between the conductors was HCONT454. This is a brand new element developed by Professor Svein Sævik, and has actually not been used much in modelling before. Whether using this element will improve the modelling of the contact between the conductors or not is still unknown at this point. \\\\The dimensions of the bend stiffener were based on very rough calculations and guidance from Professor Svein Sævik, and later optimized some what through testing and failing. Figure \ref{fig:bendstiff} shows the curvature in the bend stiffener, the "jump" at the end of the plot indicates that the bend stiffener is not perfect, and could probably have been optimized even further. 
\section{Discussion of Analyses Methodology}
\subsection{Global Analyses}
The global analyses were performed for one hour for each sea state, and rainflow counting was used to count the number of cycle sin each angle class. the classes were rearranged into 15 classes. An even more accurate result could have been obtained by increasing the number of classes to be analyzed, but at the cost of computation time in the local analyses in Bflex. 

 
\subsection{Local Analyses}
The local analyses were based on the results from the global analyses, where the 15 cases established. Due the the nature of the boundary conditions of the local model, the angle ranges from the global analyses had to be corrected as described in Section \ref{sec:localmodel}. The study to find a relation between the applied angle and total angle in bflex illustrated by Figure \ref{fig:anglecorr} shows a clear linear relation, with very good fit, although it is not perfect. From Table \ref{table:angleclass} it is clear that the angle ranges in the middle of the table are very close to each other as the classes were rearranged to inflict approximately the same damage on the cable. When correcting the angle for application in Bflex, the solution is very sensitive to small changes. If different cases were studied when attempting to find the relation, a slightly different scaling factor would have made a large impact on the results as the classes are so close together.\\\\
The angle was chosen as the master over the tension. It was assumed that the max tension occurred at max angle, min tension occurred at min angle linear,and linear interpolation was used to find the corresponding tension to each angle. This approach was very simple, and other methods could have been used. The initial plan was to look at the max and min tension that occurred for each angle range class, however, this proved to be difficult due to the nature of the Rainflow counting algorithm in Pyhton. Interpolating the tension was a simple solution, but conservative.\\\\
When the stress range was calculated form the results of the local analyses, the max tension was used for Equation \ref{eq:stressvariation2}, as this was the most conservative choice.\\\\ As the cable is covered with an outer sheath, and thus cannot be inspected for crack initiation, High Safety Class was chosen for the DFF (DFF=10.0). 
\section{Results}

\begin{table} [H]
\centering
\begin{tabular}{ |c|c|c|c|}
\hline
	Case nr & n cycles pr. year & N & damage pr year \\ 
 \hline
 \hline
	1 & 6930399.80 & 4.77E+11 & 1.45E-05 \\ 
	2 & 571967.90 & 1.97E+11 & 2.91E-06 \\ 
	3 & 237278.52 & 1.38E+11 & 1.73E-06 \\ 
	4 & 219492.01 & 9.71E+10 & 2.26E-06 \\ 
	5 & 198970.39 & 6.96E+10 & 2.86E-06 \\ 
	6 & 154052.12 & 5.00E+10 & 3.08E-06 \\ 
	7 & 130555.01 & 3.64E+10 & 3.58E-06 \\ 
	8 & 93955.72 & 2.70E+10 & 3.48E-06 \\ 
	9 & 68946.29 & 2.00E+10 & 3.45E-06 \\ 
	10 & 46839.21 & 1.49E+10 & 3.13E-06 \\ 
	11 & 54651.77 & 8.72E+09 & 6.27E-06 \\ 
	12 & 24288.68 & 5.45E+09 & 4.46E-06 \\ 
	13 & 16385.80 & 1.98E+09 & 8.27E-06 \\ 
	14 & 3355.41 & 4.00E+07 & 8.39E-05 \\ 
	15 & 139.30 & 1.16E+07 & 1.20E-05 \\ 
	\specialrule{.2em}{.1em}{.1em}
	\multicolumn{2}{c}{Total damage pr year}
&                                           
\multicolumn{2}{c}{1.56E-04} \\
	\multicolumn{2}{c}{Fatigue life}
&                                           
\multicolumn{2}{c}{6412.94 years} \\
\multicolumn{2}{c}{Fatigue life w/security}
&                                           
\multicolumn{2}{c}{641.29 years} \\
\specialrule{.2em}{.1em}{.1em} 
\end{tabular}
\caption{Fatigue life calculations for layer 2}
\label{table:fatlay2}
\end{table} 


\begin{table} [H]
\centering
\begin{tabular}{ |c|c|c|c|}
\hline
	Case nr & n cycles pr. year & N & damage pr year \\ 
 \hline
 \hline
	1 & 6930399.80 & 3.06E+12 & 2.27E-06 \\ 
	2 & 571967.90 & 1.07E+12 & 5.37E-07 \\ 
	3 & 237278.52 & 7.02E+11 & 3.38E-06 \\ 
	4 & 219492.01 & 4.68E+11 & 4.69E-07 \\ 
	5 & 198970.39 & 3.18E+10 & 6.25E-07 \\ 
	6 & 154052.12 & 2.17E+11 & 7.09E-07 \\ 
	7 & 130555.01 & 1.51E+11 & 8.63E-07 \\ 
	8 & 93955.72 & 1.07E+11 & 8.74E-07 \\ 
	9 & 68946.29 & 7.42E+10 & 9.29E-07 \\ 
	10 & 46839.21 & 5.51E+10 & 8.50E-06 \\ 
	11 & 24288.68 & 3.01E+10 & 1.18E-06 \\ 
	12 & 54651.77 & 1.79E+10 & 1.36E-06 \\ 
	13 & 16385.80 & 5.87E+09 & 2.79E-06 \\ 
	14 & 3355.41 & 8.51E+07 & 3.49E-05 \\ 
	15 & 139.30 & 2.18E+07 & 6.39E-06 \\ 
	\specialrule{.2em}{.1em}{.1em}
	\multicolumn{2}{c}{Total damage pr year}
&                                           
\multicolumn{2}{c}{6.03E-05} \\
	\multicolumn{2}{c}{Fatigue life}
&                                           
\multicolumn{2}{c}{16595.53 years} \\
\multicolumn{2}{c}{Fatigue life w/security}
&                                           
\multicolumn{2}{c}{1659.55 years} \\
\specialrule{.2em}{.1em}{.1em} 
\end{tabular}
\caption{Fatigue life calculations for layer 3}
\label{table:fatlay3}
\end{table} 
As layer 2 has the shortest fatigue life, this is the dimension layer. The fatigue life of the dynamic power cable is calculated to be 641.29 years with a safety factor of 10.0. 


