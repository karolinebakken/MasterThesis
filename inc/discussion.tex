\chapter{Results and Discussion}
\label{chap:discussion}
 In this chapter, several aspects of the project thesis will be discussed. The topic of fatigue of dynamic power cables applied offshore wind farms will be discussed based on the literature review and theory section. The global and local model will be discussed, as well as the results from the analyses. Simplifications regarding the models and analyses will also be discussed. 
 \section{Discussion of Case Study}
 \subsection{Floating Wind Turbine}
The OO-Star from Dr. Techn. Olav Olsen was chosen as the case study for floating wind turbine due to the previous familiarity with the project for the author, through a summer internship. OO-Star is part of a research project called Lifes50+ where different floating wind turbine designs and locations will be investigated. OO-Star it not fully developed, and has not yet been constructed. An advantage of choosing this concept was that there is a lot of information available on the website of Lifes50+. Another option could have been to use the Hywind concept that is already installed at the Hywind Scotland Project, or one of the other concepts in the Lifes50+ project.
\subsection{Location and Environmental Conditions}
The location of the case study was chosen to be West of Barra. This is the location in the project with the deepest water and with the most challenging weather conditions. It was thought that these factors would make the investigation of the lifetime of the power cable the most interesting in terms of fatigue. However, as the work in this study was carried out, it turned out that the wave conditions at this location was slightly too challenging, as several of the most extreme sea stats was impossible to analyze due to resonance. The solution became to move the most extreme observed sea states to sea states with smaller Tp, but still keeping the total number of observations. This way, severe resonance and extreme response was avoided. Luckily, the number of observations for the sea states that had to be modified was very low, giving little effect to the total fatigue life.\\\\
The wind conditions was significantly simplified, and only taken into concideration through the current conditions and the three different positions of the floater, Far, Neutral and Near. The offsets for the different positions were calculated very roughly and a linear system was assumed. The offsets in each position could have been studied more in depth. The scatter diagram for the wind conditions at West of Barra is available, and could have been used to simulate the wind conditions at the location more realistically, however this would make the model and the analyses more complex.


\subsection{Cable}
The cable cross section was based on guidance from Professor Svein Sævik, who has extensive experience in the field. Another approach could have been to contact a subsea cable manufacturer and either use one of their designs.  However, the majority of dynamic power cables are designed specifically for its use, and standardized models are rare. The cable proved to be some what too light weight, as the cable had large response to the vessel movements. To remedy this, 6kg of lead was added to the cross section in the space between the conductors and the inner protective sheath. This improved the inertia of the cable, but one could discuss weather double armouring should be added to the cable to increase the inertia further. Price of steel would have had to be taken into consideration for such a discussion.
\section{Discussion of Modeling Methodology}
\subsection{Global Model}
 The global model was modelled in SIMA RIFLEX. The model consists of the total cable configuration and is attached to a point a calculated distance from the center of gravity of the OO-Star. The point have its motions in accordance to the support structure of the OO-Star due to the transfer functions provided by Dr. Techn. Olav Olsen. The configuration is the characteristic lazy wave, which is simple to model, but hard to ensure that the touchdown is stable. Usually this configuration has an thether on the bottom to keep the touchdown stable, but the area of interest for this study was the cable hang off, so very little attention was paid to the bottom touchdown. The main design criteria for the configuration of the global model was that the max curvature could not be exceeded anywhere in the cable. The configuration was determined through testing and failing, and this process could have been done in a more systematic way, and the cable configuration could probably be optimized.
\subsection{Local Model}
The local model was created in BFLEX and was modelled from the inner layer and outwards. Only the very upper part of the cable was modelled as this is considered to be the area with the most fatigue damage. The friction models used to describe the contact between the layers could have been explored more and other alternative friction models could have been considered. In reality, a conductor with area 95mm$^2$ consists of 19 wires. These were not modelled, and instead equations developed by \cite{s300} was used to include the effect of the wires of the conductor. The armouring layers was only modelled with 4 wires in each layer, but this was accounted for by scaling factor. \\\\ The contact element used to describe the contact between the conductors was HCONT454. This is a brand new element developed by Professor Svein Sævik, and has actually not been used much in modelling before. Whether using this element will improve the modelling of the contact between the conductors or not is still unknown at this point. \\\\The dimensions of the bend stiffener were based on very rough calculations and guidance from Professor Svein Sævik, and later optimized some what through testing and failing. Figure \ref{fig:bendstiff} shows the curvature in the bend stiffener, the "jump" at the end of the plot indicates that the bend stiffener is not perfect, and could probably have been optimized even further. 
\section{Discussion of Analyses Methodology}
\subsection{Global Model}
The global analyses were performed for one hout each

 
\subsection{Local Model}
 
 