\chapter{Results and Discussion}
\label{chap:discussion}
 In this chapter, several aspects of the project thesis will be discussed. The topic of fatigue of dynamic power cables applied offshore wind farms will be discussed based on the literature review and theory section, and whether this topic should be investigated further in the master thesis or not. The global and local model will be discussed in modeling in SIMA RIFLEX and BFLEX and the results from testing of the model to see if they perform satisfactorily and are ready to be used in the master thesis. In addition, general simplifications and simplifications related to the models will be discussed. 
 \section{General Discussion of Topic and Case Study}
Floating offshore wind farms are often placed on locations with rough weather conditions, and damage due to fatigue are highly relevant. From the long experience with oil and gas, dynamic flexible risers and their lifetime have been investigated for decades. The motions of a floating wind turbine will be very different from a semi-submersible platform, and there is not a lot of knowledge on the lifetime of dynamic power cables applied in offshore wind. It is therefore clear that gaining more knowledge about the lifetime of dynamic power cables applied in offshore wind farm is a very interesting topic, and definitely worth investigating further in the master thesis. \\\\ When it comes to choice of a case study, the OO-Star from Dr. Techn. Olav Olsen was chosen due to the previous familiarity with the project for the author, through a summer internship. OO-Star is part of a research project called Lifes50+ where different floating wind turbine designs and locations will be investigated. OO-Star it not fully developed, and has not yet been constructed. An advantage of choosing this concept is that there is a lot of information available on the website of Lifes50+. Another option could have been to use the Hywind concept that is already installed at the Hywind Scotland Project, or one of the other concepts in the Lifes50+ project. The location of the case study was chosen to be West of Barra. This is the location in the project with the deepest water and with the most challenging weather conditions. It was thought that these factors would make the investigation of the lifetime of the power cable the most interesting in terms of fatigue. As the design is very interesting and the information about the design, the met-ocean data as well as the transfer functions are more available than for other alternatives, this has proved to be an exciting case, and OO-Star will continue to be used for the master thesis next semester. 
\section{Discussion of Modeling Methodology}
\subsection{Global Model}
 
 
\subsection{Local Model}

\section{Results}

\subsection{Global Model}


\subsection{Local Model}
 