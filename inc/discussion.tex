\chapter{Results and Discussion}
\label{chap:discussion}
 In this chapter, several aspects of the master thesis will be discussed. This includes the case study chosen, the global and local models and analyses methodologies, as well as the results from the analyses. Simplifications regarding the models and analyses will also be discussed. 
 \section{Discussion of Case Study}
 \subsection{Floating Wind Turbine}
The OO-Star from Dr. Techn. Olav Olsen was chosen as the case study for floating wind turbine due to the previous familiarity with the project for the author, through a summer internship. OO-Star is part of a research project called Lifes50+ where different floating wind turbine designs and locations will be investigated. OO-Star is not fully developed, and has not yet been constructed. An advantage of choosing this concept was that there is a lot of information available on the website of Lifes50+. Another option could have been to use the Hywind concept that is already installed at the Hywind Scotland Project, or one of the other concepts in the Lifes50+ project.
\subsection{Location and Environmental Conditions}
The location of the case study was chosen to be West of Barra. This is the location in the project with the deepest water and with the most challenging weather conditions. It was thought that these factors would make the investigation of the lifetime of the power cable the most interesting in terms of fatigue. However, as the work in this study was carried out, it turned out that the wave conditions at this location were slightly too challenging, as several of the most extreme sea stats were impossible to analyze due to severe resonance in heave. The solution became to move the most extreme observed sea states to sea states with smaller Tp, but still keeping the total number of observations the same. This way, severe resonance and extreme response was avoided. Luckily, the number of observations for the sea states that had to be modified was very low, giving little effect to the total fatigue life.\\\\
The wind conditions were significantly simplified, and only taken into consideration through the current conditions and the three different positions of the floater, Far, Neutral and Near. The offsets for the different positions were calculated very roughly and a linear system was assumed. The offsets in each position could have been studied more in depth, and more possible positions and current conditions could have been included. 


\subsection{Cable}
\label{sec:disccable}
The cable cross section was based on guidance from Professor Svein Sævik, who has extensive experience in the field. Another approach could have been to contact a subsea cable manufacturer use one of their designs.  However, the majority of dynamic power cables are designed specifically for its use, and standardized models are rare. The cable proved to be some what too light weight, as the cable had large response to the vessel movements. To remedy this, 6kg of lead was added to the cross section in the space between the conductors and the inner protective sheath. This improved the inertia of the cable, but one could discuss weather double armouring should be added to the cable to increase the inertia further. Price of steel would have had to be taken into consideration for such a discussion.
\section{Discussion of Modeling Methodology}
\subsection{Global Model}
 The global model was modeled in SIMA RIFLEX. The model consisted of the total cable configuration and is attached to a point a calculated distance from the center of gravity of the OO-Star. The point have its motions in accordance to the support structure of the OO-Star due to the transfer functions provided by Dr. Techn. Olav Olsen. The configuration was the characteristic lazy wave, which is simple to model, but hard to ensure that the touchdown is stable. Usually this configuration has an thether on the bottom to keep the touchdown stable, but the area of interest for this study was the cable hang off, so very little attention was paid to the bottom touchdown. The main design criteria for the configuration of the global model was that the max curvature could not be exceeded anywhere in the cable. The configuration was determined through an iterative process, and this process could have been done in a more systematic way. and the cable configuration could probably be optimized further.
\subsection{Local Model}
The local model was created in BFLEX and was modelled from the inner layer and outwards. Only the very upper part of the cable was modelled as this was considered to be the area with the most fatigue damage. The friction models used to describe the contact between the layers could have been explored more and other alternative friction models could have been considered. In reality, a conductor with area 95mm$^2$ consists of 19 wires. These were not modelled, and instead equations developed by \cite{s300} was used to include the effect of the wires of the conductor. The armouring layers was only modelled with 4 wires in each layer, but this was accounted for by scaling factor. \\\\The dimensions of the bend stiffener were based on very rough calculations and guidance from Professor Svein Sævik, and later optimized some what through an iterative process. Figure \ref{fig:bendstiff} shows the curvature in the bend stiffener, the "jump" at the end of the plot indicates that the bend stiffener is not perfect, and could probably have been optimized even further, event though it serves its purpuse for most cases. 
\section{Discussion of Analyses Methodology}
\subsection{Global Analyses}
The global analyses were performed for one hour for each sea state, and rainflow counting was used to count the number of cycle sin each angle class. the classes were rearranged into 15 classes. An even more accurate result could have been obtained by increasing the number of classes to be analyzed, but at the cost of computation time in the local analyses in Bflex. 

 
\subsection{Local Analyses}
The local analyses were based on the results from the global analyses, where the 15 cases were established. Due the the nature of the boundary conditions of the local model, the angle ranges from the global analyses had to be corrected as described in Section \ref{sec:localmodel}. The study to find a relation between the applied angle and total angle in Bflex illustrated by Figure \ref{fig:anglerel} shows a clear linear relation, with very good fit, although it is not perfect. From Table \ref{table:angleclass} it is clear that the angle ranges in the middle of the table are very close to each other as the classes were rearranged to inflict approximately the same damage on the cable. When correcting the angle for application in Bflex, the solution is very sensitive to small changes. If different cases were studied when attempting to find the relation, a slightly different scaling factor could have made a large impact on the results as the classes are so close together.\\\\
The angle was chosen as the master over the tension. It was assumed that the max tension occurred at max angle, min tension occurred at min angle linear, and linear interpolation was used to find the corresponding tension to each angle. This approach was very simple, and other methods could have been used. The initial plan was to look at the max and min tension that occurred for each angle range class, however, this proved to be difficult due to the nature of the Rainflow counting algorithm in Pyhton. Interpolating the tension was a simple solution, but conservative.\\\\
When the stress range was calculated form the results of the local analyses, the max tension was used for Equation \ref{eq:stressvariation2}, as this was the most conservative choice. However, a sensitivity study was conducted to investigate the effect on fatigue life if mean or min tension would have been used (see Section \ref{sec:min})\\\\ As the cable is covered with an outer sheath, and thus cannot be inspected for crack initiation, High Safety Class was chosen for the DFF (DFF=10.0). 
\section{Results}
In this section the estimated fatigue life of the power cable will be presented. This section also contains the results from the sensitivity studies, and the study of the friction mechanisms and their impact on fatigue life.

\subsection{Main Analysis of Fatigue Life}
\begin{table} [H]
\centering
\begin{tabular}{ |c|c|c|c|}
\hline
Case nr & $n_i$ pr year & $N_i$ & $d_i$ pr year \\ 
 \hline
 \hline
	1 & 6930399.80 & 4.77E+11 & 1.45E-05 \\ 
	2 & 571967.90 & 1.97E+11 & 2.91E-06 \\ 
	3 & 237278.52 & 1.38E+11 & 1.73E-06 \\ 
	4 & 219492.01 & 9.71E+10 & 2.26E-06 \\ 
	5 & 198970.39 & 6.96E+10 & 2.86E-06 \\ 
	6 & 154052.12 & 5.00E+10 & 3.08E-06 \\ 
	7 & 130555.01 & 3.64E+10 & 3.58E-06 \\ 
	8 & 93955.72 & 2.70E+10 & 3.48E-06 \\ 
	9 & 68946.29 & 2.00E+10 & 3.45E-06 \\ 
	10 & 46839.21 & 1.49E+10 & 3.13E-06 \\ 
	11 & 54651.77 & 8.72E+09 & 6.27E-06 \\ 
	12 & 24288.68 & 5.45E+09 & 4.46E-06 \\ 
	13 & 16385.80 & 1.98E+09 & 8.27E-06 \\ 
	14 & 3355.41 & 4.00E+07 & 8.39E-05 \\ 
	15 & 139.30 & 1.16E+07 & 1.20E-05 \\ 
	\specialrule{.2em}{.1em}{.1em}
	\multicolumn{2}{c}{Total damage pr year}
&                                           
\multicolumn{2}{c}{1.56E-04} \\
	\multicolumn{2}{c}{Fatigue life}
&                                           
\multicolumn{2}{c}{6412.94 years} \\
\multicolumn{2}{c}{Fatigue life w/security}
&                                           
\multicolumn{2}{c}{640.28 years} \\
\specialrule{.2em}{.1em}{.1em} 
\end{tabular}
\caption{Fatigue life calculations for layer 2}
\label{table:fatlay2}
\end{table} 


\begin{table} [H]
\centering
\begin{tabular}{ |c|c|c|c|}
\hline
Case nr & $n_i$ pr year & $N_i$ & $d_i$ pr year \\ 
 \hline
 \hline
	1 & 6930399.80 & 3.06E+12 & 2.27E-06 \\ 
	2 & 571967.90 & 1.07E+12 & 5.37E-07 \\ 
	3 & 237278.52 & 7.02E+11 & 3.38E-06 \\ 
	4 & 219492.01 & 4.68E+11 & 4.69E-07 \\ 
	5 & 198970.39 & 3.18E+10 & 6.25E-07 \\ 
	6 & 154052.12 & 2.17E+11 & 7.09E-07 \\ 
	7 & 130555.01 & 1.51E+11 & 8.63E-07 \\ 
	8 & 93955.72 & 1.07E+11 & 8.74E-07 \\ 
	9 & 68946.29 & 7.42E+10 & 9.29E-07 \\ 
	10 & 46839.21 & 5.51E+10 & 8.50E-06 \\ 
	11 & 24288.68 & 3.01E+10 & 1.18E-06 \\ 
	12 & 54651.77 & 1.79E+10 & 1.36E-06 \\ 
	13 & 16385.80 & 5.87E+09 & 2.79E-06 \\ 
	14 & 3355.41 & 8.51E+07 & 3.49E-05 \\ 
	15 & 139.30 & 2.18E+07 & 6.39E-06 \\ 
	\specialrule{.2em}{.1em}{.1em}
	\multicolumn{2}{c}{Total damage pr year}
&                                           
\multicolumn{2}{c}{6.03E-05} \\
	\multicolumn{2}{c}{Fatigue life}
&                                           
\multicolumn{2}{c}{16595.53 years} \\
\multicolumn{2}{c}{Fatigue life w/security}
&                                           
\multicolumn{2}{c}{1659.55 years} \\
\specialrule{.2em}{.1em}{.1em} 
\end{tabular}
\caption{Fatigue life calculations for layer 3}
\label{table:fatlay3}
\end{table} 
As layer 2 has the shortest fatigue life, this is the design layer. The fatigue life of the dynamic power cable is calculated to be 648.28 years with a safety factor of 10.0.
\section{Results from Sensitivity Studies and Study of Friction Mechanisms}
\subsection{Study of Sensitivity for Max, Mean and Min Tension}
When using min, mean and max tension for T in Equation \ref{eq:stressvariation2}, the fatigue life increased as shown in Table \ref{table:maxmin}

\begin{table} [H]
\centering
\begin{tabular}{ |c|c|}
\hline
T & Fatigue life [yrs] \\ 
 \hline
 \hline
	Max tension & 640.28 \\ 
	Mean tension & 881.01 \\ 
	Min tension & 1289.75 \\ 
	\hline
\end{tabular}
\caption{Fatigue life with max, mean and min tension used in Equation \ref{eq:stressvariation2}.}
\label{table:maxmin}
\end{table} 
As can be seen in Table \ref{table:maxmin}, the fatigue life increases notably when mean and min tension are used in Equation \ref{eq:stressvariation2}. The most conservative approach is naturally to use the max tension for fatigue life calculations. 
\subsection{Fatigue Life Calculated at Location with max Tension Range}
Table \ref{table:fatlay2} and \ref{table:fatlay3} are based on calculations at the location on the conductors with max curvature range (x=0.900), while \ref{table:fatlaytens2} and \ref{table:fatlaytens3} are based on calculations at the location on the conductors with max tension range (x=0.625).
\begin{table} [H]
\centering
\begin{tabular}{ |c|c|c|c|}
\hline
Case nr & $n_i$ pr year & $N_i$ & $d_i$ pr year \\ 
 \hline
 \hline
	1 & 6930399.80 & 6.73E+10 & 1.03E-4 \\ 
	2 & 571967.90 & 4.08E+10 & 1.40E-5 \\ 
	3 & 237278.52 & 3.34E+10 & 7.10E-6 \\ 
	4 & 219492.01 & 2.75E+10 & 7.98E-6 \\ 
	5 & 198970.39 & 2.27E+10 & 8.75E-6 \\ 
	6 & 154052.12 & 1.88E+10 & 8.18E-6 \\ 
	7 & 130555.01 & 1.57E+10 & 8.32E-6 \\ 
	8 & 93955.72 & 1.32E+10 & 7.14E-6 \\ 
	9 & 68946.29 & 1.11E+10 & 6.22E-6 \\ 
	10 & 46839.21 & 9.32E+09 & 5.03E-6 \\ 
	11 & 54651.77 & 6.80E+09 & 8.04E-6 \\ 
	12 & 24288.68 & 5.20E+09 & 4.67E-6 \\ 
	13 & 16385.80 & 2.93E+09 & 5.59E-6 \\ 
	14 & 3355.41 & 1.68E+08 & 2.00E-5 \\ 
	15 & 139.30 & 6.02E+07 & 2.31E-6 \\ 
	\specialrule{.2em}{.1em}{.1em}
	\multicolumn{2}{c}{Total damage pr year}
&                                           
\multicolumn{2}{c}{2.16E-04
} \\
	\multicolumn{2}{c}{Fatigue life}
&                                           
\multicolumn{2}{c}{4623.09
 years} \\
\multicolumn{2}{c}{Fatigue life w/security}
&                                           
\multicolumn{2}{c}{662.30 years} \\
\specialrule{.2em}{.1em}{.1em} 
\end{tabular}
\caption{Fatigue life calculations for layer 2 at location with max tension}
\label{table:fatlaytens2}
\end{table} 


\begin{table} [H]
\centering
\begin{tabular}{ |c|c|c|c|}
\hline
	Case nr & $n_i$ pr year & $N_i$ & $d_i$ pr year \\ 
 \hline
 \hline
	1 & 6930399.78 & 4.50E+11 & 1.54E-5  \\ 
	2 & 571967.90 & 2.49E+11 & 2.30E-6  \\
	3 & 237278.52 & 1.97E+11 & 1.20E-6  \\ 
	4 & 219492.01 & 1.57E+11 & 1.40E-6  \\ 
	5 & 198970.39 & 1.26E+11 & 1.58E-6   \\ 
	6 & 154052.13 & 1.01E+11 & 1.52E-6   \\ 
	7 & 130555.01 & 8.22E+10 & 1.59E-6  \\
	8 & 93955.72 & 6.72E+10 & 1.40E-6  \\ 
	9 & 68946.29 & 5.51E+10 & 1.25E-6  \\
	10 & 46839.21 & 4.53E+10 & 1.03E-6   \\
	11 & 54651.77 & 3.16E+10 & 1.73E-6   \\ 
	12 & 24288.68 & 2.34E+10 & 1.04E-6  \\
	13 & 16385.80 & 1.24E+10 & 1.32E-6  \\ 
	14 & 3355.41 & 5.01E+08 & 6.70E-6   \\ 
	15 & 139.30 & 1.55E+08 & 9.01E-7  \\
	\specialrule{.2em}{.1em}{.1em}
	\multicolumn{2}{c}{Total damage pr year}
&                                           
\multicolumn{2}{c}{4.04E-05
} \\
	\multicolumn{2}{c}{Fatigue life}
&                                           
\multicolumn{2}{c}{24782.03
 years} \\
\multicolumn{2}{c}{Fatigue life w/security}
&                                           
\multicolumn{2}{c}{2478.20

 years} \\
\specialrule{.2em}{.1em}{.1em} 
\end{tabular}
\caption{Fatigue life calculations for layer 3 at location with max tension}
\label{table:fatlaytens3}
\end{table}
 Both the calculations show what the dimension layer is the inner layer, with a fatigue life of 641.29 for the conductor calculated at max curvature range and a fatigue life of 662.30 for the conductor calculated at max tension range. In this case study, the most conservative estimation of fatigue life is calculated at max curvature range, but the difference is very small at only 3.28\%. 

\subsection{Fatigue Life Calculated Without Contact Between Conductors}
\begin{table} [H]
\centering
\begin{tabular}{ |c|c|c|c|}
\hline
Case nr & $n_i$ pr year & $N_i$ & $d_i$ pr year \\ 
 \hline
 \hline
1 & 6930399.80 & 7.E+11 & 8.74E-6  \\
	2 & 571967.90 & 4.78E+11 & 1.20E-6\\ 
	3 & 237278.52 & 3.92E+11 & 6.06E-7   \\ 
	4 & 219492.01 & 3.22E+11 & 6.81E-7   \\ 
	5 & 198970.39 & 2.67E+11 & 7.46E-7   \\ 
	6 & 154052.12 & 2.21E+11 & 6.98E-7 \\ 
	7 & 130555.01 & 1.83E+11 & 7.13E-7  \\ 
	8 & 93955.72 & 1.53E+11 & 6.15E-7  \\ 
	9 & 68946.29 & 1.28E+11 & 5.41E-7  \\ 
	10 & 46839.21 & 1.06E+11 & 4.40E-7  \\ 
	11 & 54651.77 & 7.55E+10 & 7.24E-7  \\ 
	12 & 24288.68 & 5.52E+10 & 4.40E-7   \\ 
	13 & 16385.80 & 2.68E+10 & 6.12E-7   \\ 
	14 & 3355.41 & 4.93E+08 & 6.81E-6   \\ 
	15 & 139.30 & 1.02E+08 & 1.37E-6   \\
	\specialrule{.2em}{.1em}{.1em}
	\multicolumn{2}{c}{Total damage pr year}
&                                           
\multicolumn{2}{c}{2.49E-05

} \\
	\multicolumn{2}{c}{Fatigue life}
&                                           
\multicolumn{2}{c}{40111.55

 years} \\
\multicolumn{2}{c}{Fatigue life w/security}
&                                           
\multicolumn{2}{c}{4011.16
 years} \\
\specialrule{.2em}{.1em}{.1em} 
\end{tabular}
\caption{Fatigue life calculations for layer 2 without effect of contact between conductors }
\label{table:fatlaycond2}
\end{table} 


\begin{table} [H]
\centering
\begin{tabular}{ |c|c|c|c|}
\hline
	Case nr & $n_i$ pr year & $N_i$ & $d_i$ pr year \\ 
 \hline
 \hline
	1 & 6930399.78 & 4.50E+11 & 1.54E-5  \\ 
	2 & 571967.90 & 2.49E+11 & 2.30E-6  \\
	3 & 237278.52 & 1.97E+11 & 1.20E-6  \\ 
	4 & 219492.01 & 1.57E+11 & 1.40E-6  \\ 
	5 & 198970.39 & 1.26E+11 & 1.58E-6   \\ 
	6 & 154052.13 & 1.01E+11 & 1.52E-6   \\ 
	7 & 130555.01 & 8.22E+10 & 1.59E-6  \\
	8 & 93955.72 & 6.72E+10 & 1.40E-6  \\ 
	9 & 68946.29 & 5.51E+10 & 1.25E-6  \\
	10 & 46839.21 & 4.53E+10 & 1.03E-6   \\
	11 & 54651.77 & 3.16E+10 & 1.73E-6   \\ 
	12 & 24288.68 & 2.34E+10 & 1.04E-6  \\
	13 & 16385.80 & 1.24E+10 & 1.32E-6  \\ 
	14 & 3355.41 & 5.01E+08 & 6.70E-6   \\ 
	15 & 139.30 & 1.55E+08 & 9.01E-7  \\
	\specialrule{.2em}{.1em}{.1em}
	\multicolumn{2}{c}{Total damage pr year}
&                                           
\multicolumn{2}{c}{4.04E-05
} \\
	\multicolumn{2}{c}{Fatigue life}
&                                           
\multicolumn{2}{c}{24782.03
 years} \\
\multicolumn{2}{c}{Fatigue life w/security}
&                                           
\multicolumn{2}{c}{2478.20

 years} \\
\specialrule{.2em}{.1em}{.1em} 
\end{tabular}
\caption{Fatigue life calculations for layer 3 without effect of contact between conductors}
\label{table:fatlaycond3}
\end{table}

\begin{table} [H]
\centering
\begin{tabular}{ |c|c|c|c|}
\hline
	Case nr.  & $\Delta T$ $N/m^2$ & $\Delta T_f$ $N/m^2$ & $\%$ of $\ Delta T$ \\ 
 \hline
 \hline
	1 & 2442.63 & 1142.47 & 0.46  \\ 
	2 & 2589.11 & 1210.25 & 0.47   \\ 
	3 & 2645.16 & 1239.32 & 0.47   \\ 
	4 & 2701.09 & 1268.36 & 0.47   \\
	5 & 2757.21 & 1297.21 & 0.47   \\ 
	6 & 2812.61 & 1326.23 & 0.47  \\ 
	7 & 2867.66 & 1353.73 & 0.47   \\ 
	8 & 2921.06 & 1379.66 & 0.47  \\ 
	9 & 2972.15 & 1404.10 & 0.47  \\ 
	10 & 3025.85 & 1429.00 & 0.47   \\ 
	11 & 3121.18 & 1469.23 & 0.47   \\ 
	12 & 3200.23 & 1492.39 & 0.47  \\ 
	13 & 3353.31 & 1507.64 & 0.45   \\ 
	14 & 3835.45 & 1101.39 & 0.29   \\ 
	15 & 3628.41 & 627.45 & 0.17   \\ 
\hline
\end{tabular}
\caption{Comparison of $\Delta T$ and $\Delta T_f$ due to contact between conductors}
\label{table:tensfri}
\end{table}
Table \ref{table:tensfri} shows that for almost all of the cases, the contact between the conductors contributes with around 47\% of the tension range. \todo{kommenter case 14 og 15}. When subtracting the tension range from the contact between the conductors, layer 2 is still the dimension layer. Excluding the effect of contact between conductors increased the fatigue life from 641.29 years to 4011.16 year, meaning it increased with 8.70 times. 

\subsection{Fatigue Life Calculated Without Friction Between Layers in Conductor}
\begin{table} [H]
\centering
\begin{tabular}{ |c|c|c|c|}
\hline
Case nr & $n_i$ pr year & $N_i$ & $d_i$ pr year \\ 
 \hline
 \hline
1 & 6930399.80 & 5.8E+12 & 1.19E-6  \\ 
	2 & 571967.90 & 1.89E+12 & 3.03E-7  \\ 
	3 & 237278.52 & 1.21E+12 & 1.95E-7   \\ 
	4 & 219492.01 & 7.91E+11 & 2.76E-7  \\ 
	5 & 198970.39 & 5.27E+11 & 3.78E-7   \\
	6 & 154052.12 & 3.52E+11 & 4.37E-7  \\ 
	7 & 130555.01 & 2.41E+11 & 5.41E-7   \\ 
	8 & 93955.72 & 1.69E+11 & 5.57E-7   \\ 
	9 & 68946.29 & 1.18E+11 & 5.83E-7 \\ 
	10 & 46839.21 & 8.39E+10 & 5.58E-7  \\ 
	11 & 54651.77 & 4.47E+10 & 1.22E-6  \\ 
	12 & 24288.68 & 2.60E+10 & 9.34E-7   \\ 
	13 & 16385.80 & 8.23E+09 & 1.99E-6   \\ 
	14 & 3355.41 & 1.06E+08 & 3.16E-5 \\ 
	15 & 139.30 & 2.63E+07 & 5.29E-6\\
	\specialrule{.2em}{.1em}{.1em}
	\multicolumn{2}{c}{Total damage pr year}
&                                           
\multicolumn{2}{c}{4.60E-05

} \\
	\multicolumn{2}{c}{Fatigue life}
&                                           
\multicolumn{2}{c}{21723.37
 years} \\
\multicolumn{2}{c}{Fatigue life w/security}
&                                           
\multicolumn{2}{c}{2172.34
 years} \\
\specialrule{.2em}{.1em}{.1em} 
\end{tabular}
\caption{Fatigue life calculations for layer 2 without effect friction between layers in conductor }
\label{table:fatlaynofri2}
\end{table} 







\begin{table} [H]
\centering
\begin{tabular}{ |c|c|c|c|}
\hline
	Case nr & $n_i$ pr year & $N_i$ & $d_i$ pr year \\ 
 \hline
 \hline
		1 & 6930399.78 & 3.05E+12 & 2.27E-6  \\ 
	2 & 571967.90 & 1.06E+12 & 5.37E-7  \\ 
	3 & 237278.52 & 7.01E+12 & 3.38E-7   \\ 
	4 & 219492.01 & 4.67E+11 & 4.70E-7  \\ 
	5 & 198970.39 & 3.18E+11 & 6.26E-7   \\ 
	6 & 154052.12 & 2.17E+11 & 7.09E-7   \\ 
	7 & 130555.01 & 1.51E+11 & 8.63E-7  \\ 
	8 & 93955.72 & 1.07E+11 & 8.74E-7  \\ 
	9 & 68946.29 & 7.66E+10 & 9.00E-7   \\ 
	10 & 46839.21 & 5.51E+10 & 8.50E-7   \\ 
	11 & 54651.77 & 3.01E+10 & 1.81E-6  \\ 
	12 & 24288.68 & 1.79E+10 & 1.36E-6   \\ 
	13 & 16385.80 & 5.87E+09 & 2.79E-6   \\ 
	14 & 3355.41 & 8.48E+07 & 3.96E-5   \\ 
	15 & 139.30 & 2.18E+07 & 6.40E-6  \\
	\specialrule{.2em}{.1em}{.1em}
	\multicolumn{2}{c}{Total damage pr year}
&                                           
\multicolumn{2}{c}{6.03E-05

} \\
	\multicolumn{2}{c}{Fatigue life}
&                                           
\multicolumn{2}{c}{16570.30

 years} \\
\multicolumn{2}{c}{Fatigue life w/security}
&                                           
\multicolumn{2}{c}{1657.03

 years} \\
\specialrule{.2em}{.1em}{.1em} 
\end{tabular}
\caption{Fatigue life calculations for layer 3 without effect of friction between layers in conductor}
\label{table:fatlaynofri3}
\end{table}
Table \ref{table:fatlaynofri2} and Table \ref{table:fatlaynofri3} shows the calculations of the fatigue life of the cable when the effect of the friction between the layers in the conductor. An interesting observation in this case is that it is the outer layer that is the design layer, with fatigue life of 1657.03 years. This is almost identical of the fatigue life calculated in the outer layer of the first calculation, displayed in Table \ref{table:fatlay3}. This shoes that the effect of friction between the layers plays a major role for the inner layer's fatigue life. When effect of friction between layers in the conductor is excluded, the fatigue life is 2.58 times higher.  



















