\chapter{Modeling Methodology}
\label{chap:procedure}
In this chapter, the methodology of the modeling process of the two models is described. The detailed properties of the two models were established in line with Professor Svein Sævik's recommendations
\section{Procedure for Modelling of Global Model}
\label{sec:globmod}
The global model was created in SIMA RIFLEX, and represented the entire dynamic part of the power cable. The cable hang off was modeled as a point with the motions of the vessel, induced by the transfer functions of OO-Star. With the assumption of most severe fatigue damage at the top of the cable, very little attention was given to the bottom touchdown. The following section describes the modeling procedure of the global model.
\subsection{Cable Configuration}
The global model consisted of 3 sections with 2 different cross-sections, as illustrated in Figure \ref{fig:globalill}.  Section 1 was the lowest part of the cable and was attached to a supernode located on the seabed, with the cable cross-section. The second section was located in the middle of the model with a buoyancy cross-section. The third section was also modeled with the cable cross-section and included the cable hang off supernode. 

\begin{figure}[H]
\centering
\includegraphics[scale=0.6]{figures/globalill}
\caption [$\; \:$ Illustration of the global model and its different sections ]{Illustration of the global model and its different sections}
 \label{fig:globalill}
\end{figure}
\noindent The main purpose of the study was to examine the contact mechanisms in the cable cross-section, rather than perform a full fatigue analysis. Consequently,  weather conditions were simplified to include 3 vessel displacements:
\begin{itemize}
    \item Neutral: There is no wind
    \item Near: Maximum wind from the east direction
    \item Far: Maximum wind from the west direction
\end{itemize}
See Section \ref{sec:current} for more detailed information on the weather conditions. \\\\
Table \ref{table:pos} show the vessel positions relative to the anchoring point. The near position was set to be 60 m away from the anchoring, in agreement with Professor Svein Sævik. The maximum offset in far position was determined to be 20\% of the water depth in the east direction, relative to the neutral position, in line with recommendations from Dr. Techn. Olav Olsen. With the assumption of a linear system, the offset in near position, relative to the neutral position was calculated from the scatter diagram for wind in Figure \ref{fig:scatterwind} as:
\begin{equation}
    \text{Offset Near}=\left(\frac{\text{Max  wind speed from east}}{\text{Max wind speed form west}}\right)^2 \cdot \text{Offset far}
\end{equation}

\begin{table} [H]
\centering
\begin{tabular}{ |c|c|}
\hline
Position & Distance from anchoring in x-direction [m] \\
 \hline
 \hline
 
Near & 60\\

Neutral & 73.25\\

Far & 96.85 \\
 
 \hline
\end{tabular}
\caption{Floater positions at different wind conditions}
\label{table:pos}
\end{table}

The design criterias for the cable configuration were no compression, and a maximum curvature of 1.3675 $\frac{1}{m}$ (\cite{API2014}) for the three vessel positions. An iterative approach was used to determine the configuration of the cable, and final measurements are displayed in Table \ref{table:DIMCABLE}.
\begin{table} [H]
\centering
\begin{tabular}{ |c|c|c|}
\hline
Section number & External area [m$^2$] & Length [m] \\
 \hline
 \hline
1 & 0.006171 & 30\\
2 & 0.03 & 40\\
3 & 0.006171 & 130\\
 \hline
\end{tabular}
\caption{Dimensions of different sections of cable}
\label{table:DIMCABLE}
\end{table}

\subsection{Cross Sectional Properties}
 Table \ref{table:crosssima} presents the properties of the two cable cross sections, and Figure \ref{fig:globalill} illustrates their use.

\begin{table} [H]
\centering
\begin{tabular}{ |c|c|c|}
\hline
Property& Cable Cross Section & Buoy Cross Section \\
 \hline
 \hline
Dry mass+ Buoyancy [$\frac{kg}{m}$] & 15.75 & 15.75\\
External Area [m]& 0.006171 & 0.03\\
Axial Stiffness [N] & 2.0e+08 & 2.0e+08\\
Bending Stiffness [Nm$^2$] & 1481 & 1481\\
Torsional Stiffness [Nm$^2$] & 5403 & 5403\\
 \hline
\end{tabular}
\caption{Parameters used for the two different cross sections}
\label{table:crosssima}
\end{table}
\noindent The bouancy elements used for section 2 of the cable was modelled with an increased external area. The values in Table \ref{table:crosssima} were calculated as follows:\newline
\newline 
\noindent The Dry mass + Buoyancy: 
\begin{equation}
\text{Dry mass + Buoyancy}=L_{c} A_cn_c (\rho_c-\rho_w) + L_{s,1} A_{s}n_{s,1} (\rho_s-\rho_w)+L_{s,2} A_{s}n_{s,2} (\rho_s-\rho_w)
\end{equation}

\noindent where $L_c$ is the length of a helical copper conductor pr meter, $A_c$ is the area of a copper conductors, $n_c$ is the number of copper conductors, $\rho_c$ is the density of copper, $\rho_w$ is the density of water, $L_{s,i}$ is the length of steel armoring pr meter for layer i, $A_s$ is the area of one wire for the steel armoring, assumed to be the same for layer 1 and 2, $n_{s,i}$ is the number of wires in armoring layer i and $\rho_s$ is the density of steel.\\\\ The plastic sheaths and tape were assumed to have the same density as the water, and thus have neutral buoyancy. \newline
\newline 
The axial stiffness and torsional stiffness were calculated according to Equation 4.14 and 4.22 in \cite{Savik2016} as:
\begin{equation}
    EA=nEA_t \cos\alpha(\cos^2\alpha-\nu_a \sin^2\alpha)
\end{equation}

\begin{equation}
    GI=nA_t E R^2 \sin^2 \alpha \cos^2 \alpha
\end{equation}
where $n$ is the number of wires in the steel armoring, $E$ is Young's modulus of steel, $A_t$ is the area of one steel wire, $\alpha$ is the lay angle of the armoring, R is the radius in polar coordinates between the armoring layers.  \newline
\newline 
The bending stiffness was calculated from the outer sheath as:
\begin{equation}
    EI= E\cdot \frac{\pi}{4}(r_o-r_i)
\end{equation}
Where E is Young's modulus of the plastic of the outer sheath, $r_o$ is the outer radius of the outer sheath, and $r_i$ is the inner radius of the outer sheath. \newline
\newline 
Lead was added to the model to increase its weight and inertia. According to Professor Svein Sævik, the following criteria would be beneficial for the cable:
\begin{equation}
    \frac{\sum_{i=1}^n m_i - b}{b}\approx 2
\end{equation}
Where $m_i$ is the mass of the different components in the cross section, and b is the buoyancy.\\\\ 6 $kg/m$ of lead was added to the model to accommodate the recommendation. It was verified that the incorporation of lead would be feasible if the center tube was made of lead, and by also adding 3 tubes identical to the center tube to the space between the conductors. As the lead did not serve any structural purpose, it was included in the model by an increase of the Dry mass + buoyancy, reaching 15.75 $\frac{kg}{m}$ as shown in Table \ref{table:DIMCABLE}. \newline
\newline An additional dummy supernode was added to the vessel with a dummy line between the two supernodes attached to the vessel. The dummy supernode enabled calculation of the angle between the vessel and cable by dot product. 

\subsection{Boundary Conditions}
The cable touchdown was fixed in all translational and rotational degrees of freedoms. The cable hang-off on the vessel was fixed in all translational degrees of freedom and free in all rotational degrees of freedom. 

\subsection{Testing of Global Model}
Prior to the analyses, the global model was verified by exposing the model to the most severe sea state and evaluating its fundamental properties. The following results for each of the three weather conditions were seen as vital to assess the quality of the model:
\begin{itemize}
    \item The static configuration without current, to make sure the cable possessed the desired steep S shape, and that the buoyancy section was not too high in the water.
    \item Curvature envelope, to make sure the curvature did not exceed the max curvature of 1.3675 $\frac{1}{m}$.
    \item Maximum force envelope, to make sure the maximum tension in the cable was tolerable.
    \item Minimum force envelope, ensure no compression in the cable.
\end{itemize}  Table \ref{table:testglob} display the verification test setup for the global model.

\begin{table} [H]
\centering
\begin{tabular}{ |c|c|c|}
\hline
Variable & Value \\
 \hline
 \hline
Hs [m] & 13.5\\
Tp[s] & 16 \\
Time step [s] & 0.1 \\
 \hline
\end{tabular}
\caption{Key variables used in testing of global model}
\label{table:testglob}
\end{table}
The results from the verification test showed no compression in the cable and acceptable curvature. The max capacity of the cable was calculated by Equation \ref{eq:tcap} to be 223kN, while the max tension in the cable during static analysis was 40.69kN, well within the tension capacity.
The results of the verification indicated that the global model performed satisfactorily. All the results were plotted and can be reviewed further in Appendix \ref{appendix:A}.
\begin{equation}
    T_{cap}=n \cos{(\alpha)} \sigma_y A_s
   \label{eq:tcap}
\end{equation}
where n is the total number of armouring layers in both layers, $\alpha$ is the lay angle, $\sigma_y$ is the yield stress of steel and $A_s$ is the area of one steel fibre.\\\\ 
 Figure \ref{fig:statcon} and Figure \ref{fig:statconc} show the global model in its three possible configurations, with and without current. 
\begin{figure}[H]
\subfloat[Near position\label{fig:cavS200}]
  {\includegraphics[width=.3\linewidth]{figures/statposnear}}\hfill
\subfloat[Neutral position \label{fig:cavS2500}]
  {\includegraphics[width=.3\linewidth]{figures/statposneu}}\hfill
  \subfloat[Far position \label{fig:cavS5000}]
  {\includegraphics[width=.3\linewidth]{figures/statposfar}}\hfill
\caption{Static configuration for the global model for different wind conditions, without current}
\label{fig:statcon}
\end{figure}

The static configuration with current for the three different wind conditions: 
\begin{figure}[H]
\subfloat[Near position\label{fig:cavS200}]
  {\includegraphics[width=.3\linewidth]{figures/statposnearc}}\hfill
\subfloat[Neutral position \label{fig:cavS2500}]
  {\includegraphics[width=.3\linewidth]{figures/statposneuc}}\hfill
  \subfloat[Far position \label{fig:cavS5000}]
  {\includegraphics[width=.3\linewidth]{figures/statposfarc}}\hfill
\caption{Static configuration for the global model for different wind conditions, with current}
\label{fig:statconc}
\end{figure}

\section{Procedure for Modelling of Local Model}
\label{sec:localmodel}
The local model was created in BFLEX and only consisted of the upper part of the cable, in line with the assumption of the location of most severe fatigue damage. The model was developed from the inner layer and outwards, with three main components, as illustrated in Figure \ref{fig:localmod}. The extra lead added to the cross section was not included due to the lack of structural importance.
\begin{figure}[H]
\centering
\includegraphics[scale=0.7]{figures/localmod.png}
\caption [$\; \:$Illustration of the local model]{Illustration of local model}
 \label{fig:localmod}
\end{figure}
\subsection{Cable}
 Figure \ref{fig:crosspro} illustrated the cable cross-section.
\begin{figure}[H]
\centering
\includegraphics[scale=0.8]{figures/cross2}
\caption [$\; \:$ Cable cross-section in the local model]{Illustration of cable cross-section in local model}
 \label{fig:crosspro}
\end{figure}
Initially, the cable model had a length of 5 pitch lengths, but this was later changed to 4.25 as it became apparent that the model was too short. The cable included several layers and components, as displayed in Figure \ref{fig:cross2}.  The straight components were modeled with HSHEAR363 elements. That includes the center tube (A), the sheath around the conductors (C), the tape between the armoring layers (E), and the outer sheath(G). The helical components were modeled with HSHEAR353 elements and included the conductors (B) and the armoring layers (D) and (F). The cable was formed out of one central node system, and all the components were connected to this node system in addition to possessing their respective radial node system.\\\\ Contact elements were used to model the interaction between the element groups. HCONT463 was used to describe the contact between different layers. The interaction between the layers was described by making one layer the master and the other the slave, and connecting their center nodes and radial nodes.  Figure \ref{fig:contact} illustrates how HCONT463 connected the different layers.

\begin{figure}[H]
\centering
\includegraphics[scale=0.5]{figures/contact}
\caption[$\; \:$ Logic of HCONT463]{The element types for the different components in the cross section and the logic of connecting them by HCONT463}
 \label{fig:contact}
\end{figure}
HCONT454 was used to describe the contact between the conductors. This element connects the two radial nodes of the HSHEAR353 elements together.\newline
\newline
The conductors were modeled with a lay angle of 8 degrees, and the armouring with a lay angle of 20 degrees. \newline
\newline
The cable included three different materials. The conductors were of copper, the armouring were of steel and the sheaths and tape were made of plastic. The material properties were decided in agreement with Professor Svein Sævik, and the most important material properties are reproduced in Table \ref{table:matprop}
\begin{table} [H]
\centering
\begin{tabular}{ |c|c|c|c|}
\hline
Property &Steel & Copper  & Plastic \\
 \hline
 \hline
Type & Linear & Linear & Elastic\\
Young's modulus [GPA] & 210 & 115 & 0.7\\
Shear modulus [GPA]& 80 & 40 &  \\
Poisson's number [-]& 0.30 & 0.36 & 0.35\\
Axial stiffness [N]& 1.48e+06 & 1.07e+07 & \\
Bending stiffness [Nm$^2$] & 0.835 & 4.19 &\\
Torsional stiffness & 0.64 & 0.32&\\
 \hline
\end{tabular}
\caption{Properties of the materials used in the local model}
\label{table:matprop}
\end{table}
\noindent The friction between layers was determined by the use of FRICONTACT materials. Table \ref{table:frimod} presents the properties of the contact materials and Table \ref{table:frimod} displays their usage. \begin{table} [H]
\centering
\begin{tabular}{ |c|c|c|c|}
\hline
Property &contact01 & contact1  & contact10 \\
 \hline
 \hline
Type & Coulomb friction & Coulomb friction & Coulomb friction\\
Static friction coeff. [-] & 0.15 & 0.15 & 0.15\\
Dynamic friction coeff. [-] & 0.15 & 0.15 & 0.15\\
Elastic stiffness in axial direction $[N/m^$^2$]$ & 100e6 & 100e6 & 100e6 \\
Elastic stiffness in transv. direction $[N/m^$^2$]$& 100e6 & 100e6 & 100e6 \\
Surface stiffness [N/m$^2$] & 100e5 & 100e6 & 100e8\\
 \hline
\end{tabular}
\caption{Properties of the friction elements used in the local model}
\label{table:friprop}
\end{table}

\begin{table} [H]
\centering
\begin{tabular}{ |c|c|}
\hline
Contact & Friction Model  \\
 \hline
 \hline
Core - Conductor & contact01\\
Conductor - Conductor & contact1\\
Conductor - Sheath & contact1\\
Sheath 1 - Armouring 1 & contact1\\
Armouring 1 - Tape &contact10\\
Tape - Armouring 2 &contact10\\
Armouring 2 - Sheath 2 & contact1\\
 \hline
\end{tabular}
\caption{Friction models for different contacts in cable}
\label{table:frimod}
\end{table}

\subsection{Bend Stiffener}
 A bend stiffener was added to the model to ease the transition between the cable and the cable hang off. The following section is based on guidance from Professor Svein Sævik. 
\newline 
\newline
The dimensions of the bend stiffener were roughly calculated early in the process with the following procedure: \newline
\newline

\noindent Locking radius of the flexible cable:
\begin{equation}
    LR = \frac{r}{1-F_j} = \frac{r}{1-0.9} = 10r
\end{equation}
Where LR is the locking radius, r is the radius of the conductor, $F_j$ is the fillfactor = 0.9 according to professor Svein Sævik. \newline
\newline
Minimum bend radius from \cite{API2014}:
\begin{equation}
   r_{min}= 1.5 * 1.1 * LR
\end{equation}
Where $r_{min}$ is the minimum locking radius, and LR is the locking radius. 
\newline
\newline
Max curvature:
\begin{equation}
   \kappa_{max}= \frac{1}{r_{min}}
\end{equation}
Where $\kappa_{max}$ is the maximum curvature, and  $r_{min}$ is the minimum locking radius.
\newline
\newline
Angle between cable and vessel: 
\begin{equation}
   \theta_{tot} = \theta_{s} + \theta_{p} +   \theta_{offset}
\end{equation}
Where $\theta_{tot}$ is the total angle at the end of the flexible cable, $\theta_{s}$ is angle due to surge, $\theta_{p}$ is angle due to pitch and $\theta_{offset}$ is the angle due to the offset of the floater. These variables are calculated as follows:\newline
\newline 
Angle due to surge:
\begin{equation}
   \theta_{s} = \arctan{(\frac{\mu_{surge}}{L_{1}})}
\end{equation}

\begin{equation}
   \mu_{surge} = \frac{H_{max}}{2} RAO
\end{equation}
Where L{1} is the assumed length from the vessel to the curve on the lazy wave configuration due to the buoyancy elements.  $H_{max}$ is the significant wave height for the most extreme sea state in the scatter diagram in Figure \ref{fig:scatn} and RAO is the response amplitude operator for surge provided by Dr. Techn. Olav Olsen. \newline
\newline 
\noindent Angle due to pitch:
\begin{equation}
   \theta_{pitch} = \frac{H_{max}}{2} RAO
\end{equation}
Where $H_{max}$ is the significant wave height for the most extreme sea state in the scatter diagram in figure \ref{fig:scatn} and RAO is the response amplitude operator for pitch provided by Dr. Techn. Olav Olsen. \newline
\newline
Angle due offset:
\begin{equation}
   \theta_{offset} = \arctan{(\frac{offset * depth}{L_1})}
\end{equation}
This can be used to calculate the length of the bend stiffener:

\begin{equation}
   L_{BS} = \frac{\theta_{max}}{\kappa_{max}} 
  \end{equation}
Where $L_{BS}$ is the length of the bend stiffener, $\theta_{max}$ is the maximum angle and $\kappa_{max}$ is the maximum curvature. \newline
\newline
Further, the outer diameter for the widest part of the bend stiffener can be calculated as follows: \newline
\newline 
Max tension is estimated to 
\begin{equation}
   T = 1.3  T_{static}
\end{equation}
Where T is the max tension, and $T_{static}$ is the static tension due to the configuration of the riser.

\begin{equation}
   T_{static} = 10 + W_s + L_1
\end{equation}
Where $W_s$ is the submerged weight of the cable and $L_1$ is the depth from the floater to the curve on the lazy wave configuration due to the buoyancy elements.\newline
\newline
\noindent $W_s$ was calculated for all the components pr meter cable the following way:

\begin{equation}
   W_s = \sum V (\rho_{comp}-\rho_{water})
\end{equation}
 Where V is the volume for each component, $\rho_{comp}$ is the density of the material of the component and $\rho_{water}$ is the density of the water.\\\\ Number of armouring fibres that fit around the cross section, the following method was used: 

 \begin{equation}
   F_j=\frac{n D}{\cos{(\alpha)} 2 \pi R}
\end{equation}
Where $F_j$ is the fill factor, n is the number of armoring fibers with diameter D  that can fit around a larger circle with radius R, and $\alpha$ is the lay angle of the armoring fibers.  \newline
\newline 
The bending stiffness EI was calculated by the following expression:

 \begin{equation}
   \kappa_{max} = \sqrt{\frac{T_{max}}{EI}}\theta_{max}
\end{equation}
Where $\kappa_{max}$ is the maximum curvature, $T_{max}$ is the maximum tension, EI is the bending stiffness, and $\theta_{max}$ is the maximum angle.\newline  
\newline 
\noindent Finally, the outer radius of the bend stiffener was calculated by:

 \begin{equation}
  EI = \frac{\pi}{4}(r_o^4 - r_i^4) 
\end{equation}
 Where EI is the bending stiffness, $r_o$ is the outer radius of the bend stiffener and $r_i$ is the inner radius.\newline 
\newline 
The bend stiffener from the rough calculations proved inadequate, and the final diemensions, showed in Table \ref{table:benddim},  were determined through an iterative process. The bend stiffener was positioned one pitchlengt from the cable termination.
 \begin{table} [H]
\centering
\begin{tabular}{ |c|c|}
\hline
Parameter & Value [m] \\
 \hline
 \hline
 
 Length & 1.400 \\
 
Outer diameter, widest part & 0.260\\

Outer diameter, narrowest part & 0.100\\

 Inner diameter & 0.097 \\
 

 \hline
\end{tabular}
\caption{Dimensions of bend stiffener}
\label{table:benddim}
\end{table}
\subsubsection{Material Data for Bend Stiffener}
The material of the bend stiffener was polyurethane with non-linear material properties. The initial Young's modulus was chosen to be 150 GPa, with decay in stiffness according to  Figure \ref{fig:matbend}.
\begin{figure}[H]
\centering
\includegraphics[scale=0.8]{figures/matbend}
\caption[$\; \:$ Stress-strain relationship for bend stiffener material]{Stress-strain relationship for bend stiffener material}
 \label{fig:matbend}
\end{figure}

\subsection{Stiff Pipe}
A stiff pipe was modeled at the end of the cable to avoid curvature and transient effects at the cable termination. The pipe had the length of one pitch length and was modeled with PIPE31 elements. The first node of the rigid pipe was connected to the first node of the central node system through the CONSTR card in BFLEX. The card assigns one node to be the slave and the other to be master, and defines the relationship between them according to Equation \ref{eq:constr}. With $C_1=1$, the nodes followed each other exactly. 
\begin{equation}
r_{Si}=C_1 \cdot r_{Mj}   
\label{eq:constr} 
\end{equation}
Where $r_{Si}$ is the displacement of the slave node,  $C_1$ is a constant describing the relationship between the master and the slave and $r_{Mj}$ is the displacement of a master node.\newline 
\newline 
\subsection{Method for Application of Tension}
\noindent For an easy application of dynamic tension to the model, an extra pipe element was added at the end of the model as can be seen in red in Figure \ref{fig:exelem}. The element had a very low EA, but the same EI as the steel armoring, enabling tension to be applied as strain so that:
\begin{equation}
    \epsilon_0 = T_0
\end{equation}
where $\epsilon_0$ is the reference value for strain and $T_0$ is the reference value for tension. This was scaled so that the reference value was $T_0=$1kN, and desired tension could easily be applied in Bflex by multiplying the strain with a factor. 
\begin{figure}[H]
\centering
\includegraphics[scale=0.8]{figures/exelem}
\caption[Illustration of the extra element added for easy application of tension]{Illustration of the extra element added for easy application of tension}
 \label{fig:exelem}
\end{figure}

\subsection{Boundary Conditions}
The boundary conditions of the local model allowed the cable to slide inside the bend stiffener. The following constraints were implemented:
\begin{itemize}
    \item Centroid system was fixed in 2 and 6 for all nodes, and 2 and 3 for the last nodes
    \item Radial nodes of the core, sheaths, and tape were fixed in 2 and 3 for all nodes. 
    \item The conductors are fixed in 1 and 2 for the first and last node. Degree of freedom 4 was fixed in all nodes.
    \item The armoring layers were fixed in 1 at the first node, and in 2,4,5 and 6 for all nodes
\end{itemize}

\subsection{Testing of Local Model}
\label{sec:localtest}
Like the global model, the local model was verified before the analyses. The following sequence of loads was added to the local model:
\begin{enumerate}
    \item Gravity Load: Gravity was excluded in the local model, but the effect of the mass of the model was applied at the first timestep. 
    \item Tension: The mean tension was applied to the local model at t=1s to t=5s, in the axial direction on the last node in the outer sheath, with a magnitude 13.7kN found from the global analysis.
    \item Initial strain: Initial strain of magnitude 0.03 was added to the outer sheath of the model in 1 direction at t=1s to t=5s.
    \item Bending: Smooth bending of the cable was applied with a dynamic angle. The cable was bent from 0 degrees to 5 degrees between t=5s and t=15s, and from 5 degrees to -5 degrees between t=15s and t=35s.  
\end{enumerate}
The following results were seen as relevant for the evaluation of the local model:
\begin{itemize}
    \item Force in the conductors over the length of the cable
    \item Moments about X, Y and Z axis over the length of the cable
    \item Curvature of the outer sheath over the length of the cable
    \item Curvature in bend stiffener
\end{itemize}
The testing revealed that the local model seemed acceptable. All the results of the testing of the local model are in Appendix \ref{appendix:B}. Figure \ref{fig:bflexax} shows that the tension variation on the conductors between max angle and min angle died out towards the end of the model, indicating that the model was of sufficient length, and the same can be seen for the moments in Figure \ref{fig:bflexmx} to Figure \ref{fig:bflexmz}. The curvature was about zero at the beginning and end of the outer sheath (Figure \ref{fig:bflexcurve}), and the curvature in the bend stiffener (Figure \ref{fig:bendstiff}) has the desired shape. The results indicated that the bend stiffener had appropriate dimensions and was serving its purpose. \\\\
Figure \ref{fig:local1} and Figure \ref{fig:local2} show the local model and some of its components as they appeared in XPOST.

\begin{figure}[H]
\subfloat[Local model from the side with bend stiffener \label{fig:lm_total}]
  {\includegraphics[width=.45\linewidth]{figures/lm_total}}\hfill
\subfloat[Cable cross section in local model \label{fig:lm_cross}]
  {\includegraphics[width=.25\linewidth]{figures/lm_cross}}\hfill
\caption{Local model}
\label{fig:local1}
\end{figure}

\begin{figure}[H]
\subfloat[Conductors \label{fig:single}]
  {\includegraphics[width=.45\linewidth]{figures/lm_conductors}}\hfill
\subfloat[Armouring \label{fig:3phase}]
  {\includegraphics[width=.45\linewidth]{figures/lm_arm}}\hfill
\caption{Helical components in the local model}
\label{fig:local2}
\end{figure}





 



