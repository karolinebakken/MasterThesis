\chapter*{Sammendrag}
De siste årene har utviklingen innenfor offshore vind teknologi vært at installasjonene plasseres leng unna kysten med dypere vann. De dynamiske kraftkablene som leverer strømmen fra installasjonene er viktige komponenter for denne typen teknologi. Mye forskning har blitt gjort på slanke konstruksjoner i forbindelse med olje og gass industrien, men svært lite er blitt gjort med tanke på utmatting i strømkabler anvendt i flytende offshore vind teknologi. En strømkabel består vanligvis av ledere med tråder som er orndet i spiraler rundt en kjærne. Dette skaper kontakt mellom både de ulike lederene, men også mellom lagene av tråder i hver enkelt leder, og gjør kablene sårbare for ulike utmattingsmekanismer. \\\\
Ved å utføre globale og lokale analyser ble ulike kontaktmekanismer i den dynamiske strømkabelen undersøkt. Den flytende vindturbinen OO-Star ble ansett som en interessant konstruksjon å bruke til dette prosjektet. OO-Star er et design utviklet av Dr. Techn. Olav Olsen som en del av forskningsprosjektet Lifes50+. West of Barra utenfor Skotland ble brukt som lokasjon, og kabeldesignet ble utformet i nøye dialog med Professor Svein Sævik. SIMA RIFLEX ble brukt for å modellere den globale modellen som består av hele kabelen. Den ble festet til havbunnen og til et punkt på overflaten som beveger seg likt som OO-Star, gjennom transfer funksjonene fra Dr. Techn. Olav Olsen. Kabelkonfigurasjonen ble basert på kravet om ingen kompressjon i kabelen, samt en maks kurvatur som ikke kunne overskrides. Kun tre vindtilstander med tilhørende flyterposisjoner ble brukt som en forenkling. Den lokale modellen bestod av den øvre delene av kabelen, ettersom det ble ansett som sannsynlig at utmattingsskadene ville være størst her. Den lokale modellen bestod av kabelen selv med dens mange lag, et stivt rør og en bøyestiver.\\\\
Den globale analysen be utført ved at alle sjøtilstandene beskrevet i scatter tabellen til West of Barra ble analysert i en time hver. Vinkelen mellom turbinkonstruksjonen og den øvre delen av kabelen ble regnet ut for hvert tidssteg, samt spenningen i det øverste elementet av kabelen. Rainflow Counting ble brukt for å telle antall sykluser for hver vinkelklasse, som ble reorganisert inn til 15 klasser som påførte ca samme utmattingsskade på kabelen. Dette ble brukt som input til den lokale analysen, der hver klasse med vinkel og tilhørende spenning ble analysert i en syklus. Den resulterende kurvaturen og spenningen for hver klasse ble brukt sammen med den analytiske modellen til å regne ut spenningsintervallet for hver klasse. Deretter ble utmattigslevetiden regnet ut fra en passene SN-kurve. \\\\
Utmattingslevetiden til den dynamiske strømkabelen ble regnet ut til å være 267.37 år. I tilegg ble konkludert med at kontakt mellom ledere, og kontakt mellom lagene i hver leder er av høy betydning for utmattinslevetiden. 

