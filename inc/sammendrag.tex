\chapter*{Sammendrag}
Utviklingen i offshore vind tenkologi er lenger unna kysten og mot dypere vann, noe som gjøre at flytende installasjoner er det mest gjennomførbare. De dynamiske kraftkablene som leverer strømmen fra installasjonene er viktige komponenter for denne typen teknologi. Mye forskning har blitt gjort på slanke konstruksjoner i forbindelse med olje og gass industrien, men svært lite er blitt gjort med tanke på utmatting i strømkabler anvendt i flytende offshore vind teknologi. En strømkabel består vanligvis av ledere med en samling tråder som er orndet i spiraler rundt en kjærne. Dette skaper kontakt mellom både de ulike lederene, men også mellom lagene av tråder i hver enkelt leder, og gjør kablene sårbare for ulike utmattingsmekanismer. \\\\
Dette temat ble undersøke nærmere ved å utføre både lokale og globale analyser. Den flytende vindturbinen OO-Star ble ansett som en interessant konstruksjon å bruke til dette prosjektet. OO-Star er et design utviklet av Dr. Techn. Olav Olsen som en del av forskningsprosjektet Lifes50+. West of Barra utenfor Skotland ble brukt som lokasjon, og kabeldesignet ble utformet i nøye dialog med Professor Svein Sævik. SIMA RIFLEX ble brukt i for å modellere den globale modellen som består av hele kabelen, festet til havbunnen og til et punkt på overflaten som beveger seg likt OO-Star gjennom transfer funksjonene gitt av Dr. Techn. Olav Olsen. Kabelkonfigurasjonen ble basert på kravet om ingen kompressjon i kabelen, samt en maks kuratur som ikke kunne overskrides. Kun tre vind tilstander med tilhørende flyterposisjoner ble brukt som en forenkling. Den lokale modellen bestod av den øvre delene av kabelen, ettersom det ble ansett som mest sannsynlig at utmattingsskadene ville være størst her. Den lokale modellen bestod av kabelen selv med dens mange lag, et stivt rør og en bøyestiver.\\\\
Den globale analysen be utført ved at alle sjøtilstandene beskrevet i scatter tabellen til West of Barra ble analysert i en time hver. Vinkelen mellom turbinkonstruksjonen og den øvre delen av kabelen ble regnet ut for hvert tidssteg, samt spenningen i det øverste elementet av kabelen. Rainflow counting ble brukt for å telle antall sykluser for hver vinkelklasse, som ble reorganisert inn til 15 klasser med vinkel og spenning for hver klasse. Dette ble brukt som input til den lokale analysen, der hver klasse  med vinkel og spenning ble analysert i en syklus. Den resulterende kurvaturen og spenningen for hver klasse ble brukt til å regne ut belastningsområdet for hver klasse og utmattigslevetiden ble regnet ut fra en passene SN-kurve. \\\\
Utmattingslevetiden til den dynamiske strømkabelen ble regnet ut til å være 640.28 år, og det ble konkludert med at det var stor effekt av frisksjon mellom lederene samt mellom lagene internt i hver leder. \\\\
